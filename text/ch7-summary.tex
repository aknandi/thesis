\clearpage
%\begin{savequote}[8cm]
%\textlatin{Neque porro quisquam est qui dolorem ipsum quia dolor sit amet, consectetur, adipisci velit...}
%
%There is no one who loves pain itself, who seeks after it and wants to have it, simply because it is pain...
%  \qauthor{--- Cicero's \textit{de Finibus Bonorum et Malorum}}
%\end{savequote}

\chapter{\label{ch:7-summary}Conclusion} 

\section{Summary of results}

A study of \CP violation in the \btodkst mode was performed, with \D meson decays to \Kp\pim, \Kp\pim, \pip\pim, \Kp\pim\pip\pim and \pip\pim\pip\pim final states. The data analysed in this study was collected by the \lhcb detector from proton-proton collisions at the Large Hadron Collider (LHC) between 2011 and 2016. This study used 1\invfb and 2\invfb of $pp$ collisions at centre of mass energy, $\sqrt{s} = 7\tev \text{ and } 8\tev$ collected in 2011 and 2012, and 1.8\invfb at $\sqrt{s} = 13\tev$ collected in 2015 and 2016. An GLW/ADS analysis was carried out to extract the \CP observables by performing a simultaneous fit to the data using a model developed to describe the selected \btodkst decays.

The world's most precise measurements of \CP violation in \btodkst decays were made using approximately 3660 events across all \Dz decay modes. The results are from this analysis are:

\begin{alignat*}{25}
A_{K\pi} &= &\ -&0.004&\ &\pm&\ &0.023&\ &\pm&\ &0.008& \qquad\qquad
A_{K\pi\pi\pi} &= &\ -&0.013&\ &\pm&\ &0.031&\ &\pm&\ &0.009& \\
A_{KK} &= &&0.06&\ &\pm&\ &0.07&\ &\pm&\ &0.01& 
A_{\pi\pi\pi\pi} &= &&0.02&\ &\pm&\ &0.11&\ &\pm&\ &0.01& \\
A_{\pi\pi} &= &&0.15&\ &\pm&\ &0.13&\ &\pm&\ &0.02& 
R_{\pi\pi\pi\pi} &= &&1.08&\ &\pm&\ &0.13&\ &\pm&\ &0.03& \\
R_{KK} &= &&1.22&\ &\pm&\ &0.09&\ &\pm&\ &0.01& 
R^+_{K\pi\pi\pi} &= &&0.016&\ &\pm&\ &0.007&\ &\pm&\ &0.003& \\
R_{\pi\pi} &= &&1.08&\ &\pm&\ &0.14&\ &\pm&\ &0.03& 
R^-_{K\pi\pi\pi} &= &&0.006&\ &\pm&\ &0.006&\ &\pm&\ &0.004& \\
R^+_{K\pi} &= &&0.020&\ &\pm&\ &0.006&\ &\pm&\ &0.001& &&&&&&&&&&&&\\ 
R^-_{K\pi} &= &&0.002&\ &\pm&\ &0.004&\ &\pm&\ &0.001& &&&&&&&&&&&&
\end{alignat*} 
%\begin{alignat*}{13}
%A_{K\pi} &= &\ -&0.004&\ &\pm&\ &0.023&\ &\pm&\ &0.008& \\
%A_{KK} &= &&0.06&\ &\pm&\ &0.07&\ &\pm&\ &0.01& \\
%A_{\pi\pi} &= &&0.15&\ &\pm&\ &0.13&\ &\pm&\ &0.02& \\
%R_{KK} &= &&1.22&\ &\pm&\ &0.09&\ &\pm&\ &0.01& \\
%R_{\pi\pi} &= &&1.08&\ &\pm&\ &0.14&\ &\pm&\ &0.03& \\
%R^+_{K\pi} &= &&0.020&\ &\pm&\ &0.006&\ &\pm&\ &0.001& \\ 
%R^-_{K\pi} &= &&0.002&\ &\pm&\ &0.004&\ &\pm&\ &0.001& \\
%A_{K\pi\pi\pi} &= &\ -&0.013&\ &\pm&\ &0.031&\ &\pm&\ &0.009& \\
%A_{\pi\pi\pi\pi} &= &&0.02&\ &\pm&\ &0.11&\ &\pm&\ &0.01& \\
%R_{\pi\pi\pi\pi} &= &&1.08&\ &\pm&\ &0.13&\ &\pm&\ &0.03& \\
%R^+_{K\pi\pi\pi} &= &&0.016&\ &\pm&\ &0.007&\ &\pm&\ &0.003& \\ 
%R^-_{K\pi\pi\pi} &= &&0.006&\ &\pm&\ &0.006&\ &\pm&\ &0.004&
%\end{alignat*}
The measurements are found to consistent and more precise compared to previous measurements from \babar~\cite{BaBarDKstar}. The first evidence for \CP violation in the two-body ADS mode was obtained with a significance of 4.2$\sigma$. The senstivity to \Pgamma was extracted, as well as obtaining the current best sensitivity to the hadronic parameters of the \Bm decay, namely \rb and \deltab. CL on \Pgamma.

\section{Outlook}

Understanding \CP violation in the quark sector of the Standard Model, through the CKM matrix, is of key importance in particle physics. The unitarity of the CKM matrix can be tested via the comparison of the CKM angle \Pgamma between direct and indirect measurements. The world's best direct measurement, from studies of \CP violation in various tree-level \B meson decays using the \lhcb detector, is $\Pgamma = \left(72.2^{+6.8}_{-7.3}\right)^{\circ}$~\cite{LHCb-PAPER-2016-032}, where the uncertainty can be reduced by using more data and including a wider range of decay modes. Indirect determination of \Pgamma produces a measurements of $\Pgamma = (65.3^{+1.0}_{-2.5})^{\circ}$~\cite{CKMFitter}, which is obtained from a global fit to the CKM triangle, excluding all direct measurements. The uncertainties on this measurements are dominated by lattice QCD. Currently, direct and indirect measurements of \Pgamma are consistent with each other. Inconsistencies between these measurements would suggest New Physics via non-unitarity of the CKM matrix. Reducing the uncertainty on direct measurements of \Pgamma by analysing more data and more decay modes may reveal inconsistencies.

A key aim of the \lhcb experiment is to continue to improve precision on the determination of \Pgamma from direct measurements, which is limited by statistical uncertainty. Therefore, \lhcb aims to reduce the uncertainty in this measurement by analysing more data and more decay modes.  To date, the \lhcb detector has collected 1\invfb and 2\invfb of $pp$ collisions at centre of mass energy, $\sqrt{s} = 7\tev \text{ and } 8\tev$ in 2011 and 2012, and 3.4\invfb at $\sqrt{s} = 13\tev$ in 2015, 2016 and 2017. Another 1.6\invfb is expected to be collected at $\sqrt{s} = 13\tev$ in 2018, totalling 5\invfb of $\sqrt{s} = 13\tev$ in Run 2 (from 2015 to 2018). In this thesis, it was found that the \btodkst yield obtained per unit of integrated luminosity increased by a factor of 3 from \runone to \runtwo, mainly due to the increase in centre of mass energy from $\sqrt{s} = 8\tev$ to $\sqrt{s} = 13\tev$. Other similar \Pgamma-sensitive modes have shown an factor of 2-3 increase in the yield per unit of integrated luminosity~\cite{LHCb-PAPER-2017-021}. Therefore, the expected sensitivity to \Pgamma at the end of Run 2 is $4^{\circ}$ Ref. After Run 2, there is a break in data-taking, where the detector will be upgraded, allowing it to operate at a instantaneous luminosity of $2 \times 10^{33} cm^{-2}s^{-1}$, due to the change to a fully software-based trigger, and centre of mass energy of $\sqrt{s} = 14\tev$. This upgrade results in a predicted luminosty of 50\invfb by the end of 2030, and an increase in signal efficiency. This leads to a predicted sensitivity to \Pgamma of $1^{\circ}$ by the end of 2030 Ref. The Belle II experiment, which is expecting to collect a data sample corresponding to an integrated luminosity of 50$ab^{\text{-1}}$ between 2018 and 2026, predicts a determination of \Pgamma to $1^{\circ}$ or better~\cite{BelleII}.

This thesis presented measurements of \CP observables in the \btodkst mode for the first time at \lhcb. The ADS/GLW methods implemented have been widely used in previous \Pgamma-sensitive measurements~\cite{LHCb-PAPER-2017-021,LHCb-PAPER-2016-006}. These \Dz decay modes have a reasonably high branching ratio and are relatively simple to interpret. The results from this thesis, when combined with other \Pgamma-senstive decays, will further constrain the presion of \Pgamma from direct measurements. Another commonly used method, called the GGSZ method~\cite{GGSZ}, involves analysing \D decays to \KS\pip\pim and \KS\Kp\Km final states~\cite{LHCb-PAPER-2014-041}. A key advantage of the GGSZ method is its ability to resolve the two-fold ambiguity of \Pgamma obtained with from the ADS/GLW method. 

The angle \Pgamma represents a way of parameterising \CP violation in the Standard Model. Inconsistencies between direct and indirect measurements of \Pgamma could indicate new \CP violation effects beyond the Standard Model. This may open the door towards a deeper understanding of the large matter-antimatter asymmetry observed in the universe.
