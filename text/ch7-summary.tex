\clearpage

\chapter{\label{ch:7-summary}Conclusion} 

\section{Summary of results}

A study of \CP violation in the \btodkst mode was performed, with \D meson decays to \Kp\pim, \Kp\pim, \pip\pim, \Kp\pim\pip\pim and \pip\pim\pip\pim final states. The data analysed in this study were collected by the \lhcb detector from proton-proton collisions at the Large Hadron Collider (LHC) between 2011 and 2016. This study used 1\invfb and 2\invfb of $pp$ collisions at centre of mass energy, $\sqrt{s} = 7\tev \text{ and } 8\tev$ collected in 2011 and 2012, and 1.8\invfb at $\sqrt{s} = 13\tev$ collected in 2015 and 2016.

The world's most precise measurements of \CP violation in \btodkst decays were made using approximately 3660 signal events across all \Dz decay modes. The results are from this analysis are:

\begin{alignat*}{25}
A_{K\pi} &= &\ -&0.004&\ &\pm&\ &0.023&\ &\pm&\ &0.008& \qquad\qquad
A_{K\pi\pi\pi} &= &\ -&0.013&\ &\pm&\ &0.031&\ &\pm&\ &0.009& \\
A_{KK} &= &&0.06&\ &\pm&\ &0.07&\ &\pm&\ &0.01& 
A_{\pi\pi\pi\pi} &= &&0.02&\ &\pm&\ &0.11&\ &\pm&\ &0.01& \\
A_{\pi\pi} &= &&0.15&\ &\pm&\ &0.13&\ &\pm&\ &0.02& 
R_{\pi\pi\pi\pi} &= &&1.08&\ &\pm&\ &0.13&\ &\pm&\ &0.03& \\
R_{KK} &= &&1.22&\ &\pm&\ &0.09&\ &\pm&\ &0.01& 
R^+_{K\pi\pi\pi} &= &&0.016&\ &\pm&\ &0.007&\ &\pm&\ &0.003& \\
R_{\pi\pi} &= &&1.08&\ &\pm&\ &0.14&\ &\pm&\ &0.03& 
R^-_{K\pi\pi\pi} &= &&0.006&\ &\pm&\ &0.006&\ &\pm&\ &0.004& \\
R^+_{K\pi} &= &&0.020&\ &\pm&\ &0.006&\ &\pm&\ &0.001& &&&&&&&&&&&&\\ 
R^-_{K\pi} &= &&0.002&\ &\pm&\ &0.004&\ &\pm&\ &0.001& &&&&&&&&&&&&
\end{alignat*} 
These measurements are found to consistent and more precise than the previous measurements from \babar~\cite{BaBarDKstar}. The first evidence of the two-body ADS mode was obtained with a signal significance of 4.2$\sigma$. The sensitivity to \Pgamma was extracted, obtaining a central value of $40.9^{\circ}$ with a $1\sigma$ confidence interval of $[24.7, 60.2]^{\circ}$, and $2\sigma$ confidence intervals of $[10.2, 86.1]^{\circ}$ and $[100.0,165.2]^{\circ}$. Also, the current best sensitivity to the hadronic parameters of the \Bm decay, namely \rb and \deltab, was achieved.

The work presented in this thesis provides a contribution to the measurement of \Pgamma. The angle \Pgamma represents a way of parameterising \CP violation in the Standard Model. Inconsistencies between direct and indirect measurements of \Pgamma could indicate new \CP violation effects beyond the Standard Model. This may open the door towards a deeper understanding of the large matter-antimatter asymmetry observed in the universe.
