\clearpage
%\begin{savequote}[8cm]
%\textlatin{Neque porro quisquam est qui dolorem ipsum quia dolor sit amet, consectetur, adipisci velit...}
%
%There is no one who loves pain itself, who seeks after it and wants to have it, simply because it is pain...
%  \qauthor{--- Cicero's \textit{de Finibus Bonorum et Malorum}}
%\end{savequote}

\chapter{\label{ch:6-interpretation}Extraction of CKM angle \Pgamma} 

\minitoc

The \CP observables from \btodkst decays are used to determine the physics parameters, \rb, \deltab and \Pgamma, via Eqs.~\ref{exp_Acp} - \ref{exp_Rpm4body}. In this determination, other parameters that appear in Eqs.~\ref{exp_Acp} - \ref{exp_Rpm4body} are taken directly from previous measurements, given in Table \ref{inputparameters}, to be used as external inputs. The coherence factor $\kappa$, discussed in Section \ref{sec:theory:gamma}, is estimated, as described in Section \ref{sec:interpretation:coherence}, and used as an extra constraint when determining \rb, \deltab and \Pgamma. The variations in acceptance across the four-body phase space and the effect of this on the interpretation is considered in Section \ref{sec:interpretation:inputs}. Section \ref{sec:interpretation:gammadini} discusses the determination of the physics parameters: \rb, \deltab and \Pgamma from measurements of \btodkst decays. Finally, Section \ref{sec:interpretation:futuresensitivity} discusses the expected sensitivity of the \btodkst channel to \rb, \deltab and \Pgamma, with an increased dataset after further running periods of the LHC.

\section{Coherence factor, $\kappa$}
\label{sec:interpretation:coherence}

Due to the large natural width of the \Kstarm meson, in the region near the \Kstarm mass interference may occur between the signal \Kstarm decay amplitude and amplitudes due to other \decay{\Bm}{\D\KS\pim} contributions, for example higher \KS\pim resonances and non-resonant decays. The presence of these interfering contributions when analysing the \btodkst decays dilutes the sensitivity to \Pgamma, which is quantified by a coherence factor, $\kappa$, where $0 \leq \kappa \leq 1$ and $\kappa = 1$ denotes a pure \Kstarm contribution, giving maximum sensitivity to \Pgamma. The coherence factor, $\kappa$, is discussed in more detail in Section \ref{sec:theory:gamma}. 

In this section the coherence factor, $\kappa$, is estimated in order to be used as an extra input, along with the \CP observables, in the extraction of \rb, \deltab and \Pgamma. The coherence factor $\kappa$ is estimated by developing an amplitude model for the \decay{\Bm}{\D\KS\pim} decay, and then this model and the definition of $\kappa$, given in Equation \ref{kappadefinition}, are used to estimate $\kappa$ in this analysis.

\subsection{Decay model}
\label{sec:interpretation:model}

An amplitude model of \decay{\Bm}{\D X^-} decays is developed, where $X^-$ represents a resonant or non-resonant \KS\pim pair. The model contains resonant and non-resonant decays that are expected to contribute to in the region of phasespace used in this analysis, defined by the \Kstarm mass and \KS helicity angle requirements. The components of the model used for this study are:

\begin{itemize}
\item $\decay{\Bp}{\Dz K^*(892)^+}$ and $\decay{\Bp}{\Dzb K^*(892)^+}$
\item $\decay{\Bp}{\Dz K^*_0(1430)^+}$ and $\decay{\Bp}{\Dzb K^*_0(1430)^+}$
\end{itemize}

The model for the $K^*_0(1430)^+$ uses a parameterisation developed by the LASS experiment~\cite{LASS}, which approximately consists of a relativistic Breit-Wigner component corresponding to the resonant $K^*_0(1430)^+$ and a non-resonant scattering component. Other resonances e.g. $K^*(1680)^+$ and $D_2^*(2460)^-$, are considered to be negligible in the region being considered and so are not included in the decay model. The parameters of the resonances are listed in Table \ref{resonances}.

\begin{table}[h]
\centering
\begin{tabular}{llll}
\hline
Resonance & Mass, M \mev & Width, $\Gamma$ \mev & Spin \\
\hline
$K^*(892)^+$ & $891.66 \pm 0.26$ & $50.8 \pm 0.9$ & 1 \\
$K^*_0(1430)^+$ & $1425 \pm 50$ & $270 \pm 80$ & 0 \\
\hline
\end{tabular}
\caption{Parameters of the resonances in the decay model.}
\label{resonances}
\end{table}

In order to calculate $\kappa$, it is necessary to consider the magnitudes and phases of the model components described. The parameters $A_{\uquark\bquark}$, containing \decay{\bquark}{\uquark} transitions, and $A_{\cquark\bquark}$, containing \decay{\bquark}{\cquark} transitions, are the amplitude of the suppressed and favoured \decay{\Bm}{\D X^-} decays, where $X^-$ represents a resonant or non-resonant \KS\pim pair. The amplitudes $A_{\uquark\bquark}$ and $A_{\cquark\bquark}$ are modelled as,
%The amplitude model was generated using the Laura++ package~\cite{Laura}.
\begin{align*}
A_{\uquark\bquark} =\ & a_{\uquark\bquark}^{K^*(892)^+}e^{-i\delta_{\uquark\bquark}^{K^*(892)^+}}RelBW(p;M_{K^*(892)^+},\Gamma_{K^*(892)^+},1)\ + \\
& a_{\uquark\bquark}^{K^*_0(1430)^+}e^{-i\delta_{\uquark\bquark}^{K^*_0(1430)^+}}LASS\_BW(p;M_{K^*_0(1430)^+},\Gamma_{K^*_0(1430)^+},0)\ + \\
& a_{\uquark\bquark}^{NR}e^{-i\delta_{\uquark\bquark}^{NR}}LASS\_NR
\end{align*}
and,
\begin{align*}
A_{\cquark\bquark} =\ & a_{\cquark\bquark}^{K^*(892)^+}e^{-i\delta_{\cquark\bquark}^{K^*(892)^+}}RelBW(p;M_{K^*(892)^+},\Gamma_{K^*(892)^+},1)\ + \\
& a_{\cquark\bquark}^{K^*_0(1430)^+}e^{-i\delta_{\cquark\bquark}^{K^*_0(1430)^+}}LASS\_BW(p;M_{K^*_0(1430)^+},\Gamma_{K^*_0(1430)^+},0)\ + \\
& a_{\cquark\bquark}^{NR}e^{-i\delta_{\cquark\bquark}^{NR}}LASS\_NR
\end{align*}
where, $RelBW$, $LASS\_BW$ and $LASS\_NR$ refer to the relativistic Breit-Wigner, LASS Breit-Wigner and LASS non-resonance shapes~\cite{LASS}. The arguments of a given resonance $R(p;M,\Gamma,J)$ correspond to the position in \decay{\Bm}{\D\KS\pim} phasespace ($p$), and the mass ($M$), width ($\Gamma$) and spin ($J$) of the resonance $R$.% as implemented in Laura++~\cite{Laura}. 

The relative square of the magnitude of the various components is equal to the relative branching fractions in the limit of no interference. This scenario of no interference is assumed. The only available branching fraction measurement is,
\begin{equation*}
\BF(\decay{\Bp}{\Dzb K^*(892)^+}) \times \BF(\decay{K^*(892)^+}{\KS\pip}) = 1.8 \times 10^{-4}
\end{equation*}

It is assumed that the different resonant \Kstarp modes are produced with the same branching fraction as the $K^*(892)^+$ mode, e.g $\BF(\decay{\Bp}{\Dzb K^*(892)^+}) = \BF(\decay{\Bp}{\Dzb K^*(1430)^+})$. The branching fractions of different resonant \Kstarp modes to the \KS\pip final state are also taken into account, namely $\BR(\decay{K^*(892)^+}{\KS\pip}) = \frac{1}{3}$ and $\BR(\decay{K^*_0(1430)^+}{\KS\pip}) = 0.31$.

The non-resonant branching fraction \BR(\decay{\Bp}{\Dzb\KS\pip}) is not known. Therefore, assuming,
\begin{equation*}
\frac{\BR(\decay{\Bz}{\Dm\Kz\pip})}{\BR(\decay{\Bz}{\Dm\Kzb\Kp})} = \frac{\BR(\decay{\Bp}{\Dzb\Kz\pip})}{\BR(\decay{\Bp}{\Dzb\Kzb\Kp})}
\end{equation*}
and using measurements and upper limits for the other branching fractions~\cite{PDG2014}, the value of \BR(\decay{\Bp}{\Dzb\KS\pip}) is estimated to have an upper limit of $(5.2 \pm 0.3) \times 10^{-4}$.

When generating an amplitude model, only the relative amplitudes and phases of the various components are needed, therefore the $K^*(892)$ is fixed to have an amplitude of 1 and phase of 0. The relative amplitude $r_B$ is assumed to be 0.1. Using the estimates for the various branching ratios involved in the model, the values of the squares of the amplitudes $a_{\uquark\bquark}$ and $a_{\cquark\bquark}$ of the model components are taken in the ranges:

\begin{itemize}
\item \textbar $a_{\cquark\bquark}^{K^*(892)^+}$\textbar$^2 = 1$, \hspace{12pt} with $a_{\uquark\bquark}^{K^*(892)^+} = r_B a_{\cquark\bquark}^{K^*(892)^+}$
\item \textbar $a_{\cquark\bquark}^{K^*_0(1430)^+}$\textbar$^2 \in [0.7 \times 0.93,1.3 \times 0.93]$, \hspace{14pt} with $a_{\uquark\bquark}^{K^*_0(1430)^+} = r_B a_{\cquark\bquark}^{K^*_0(1430)^+}$
\item \textbar $a_{\cquark\bquark}^{NR}$\textbar$^2 \in [0.0,2.4]$, \hspace{12pt} with $a_{\uquark\bquark}^{NR} = r_B a_{\cquark\bquark}^{NR}$
\end{itemize}

%and,
%
%\begin{itemize}
%\item $a_{\uquark\bquark}^{K^*(892)^+} = r_B a_{\cquark\bquark}^{K^*(892)^+}$
%\item $a_{\uquark\bquark}^{K^*_0(1430)^+} = r_B a_{\cquark\bquark}^{K^*_0(1430)^+}$
%\item $a_{\uquark\bquark}^{NR} = r_B a_{\cquark\bquark}^{NR}$
%\end{itemize}

Figure \ref{dalitzplot} shows an example of the amplitude model described. The \Kstar selection in this analysis, consisting of the \Kstar mass and \KS helicity angle requirements, are represented by dashed lines in this plot.

\begin{figure}[h]
\centering
\includegraphics[width=0.8\linewidth]{figures/results/dalitz.pdf}
\caption{An example amplitude model used in the estimate of $\kappa$. The horizontal axis labelled $M_{D\pi}^2$ is defined as $(p_D + p_{\pi})^2$, where $p_{X}$ is the four-momentum of particle $X$. Similarly, the vertical axis, labelled $M_{K_s\pi}^2$, is defined as $(p_{K_s} + p_{\pi})^2$. The projections in these two coordinates are shown. The $K^*(892)^+$ and $K^*(1430)^+$ resonances can be clearly seen in the red projection on the right hand side of the figure. The dashed lines on the plot represent the \Kstar mass and \KS helicity angle selection used in this analysis.}
\label{dalitzplot}
\end{figure}

\subsection{Estimation of the coherence factor, $\kappa$}
\label{sec:interpretation:kappa}

One thousand samples are generated based on the amplitude model described in Section \ref{sec:interpretation:model}. For each sample the amplitudes and phases of the different resonances are varied within limits; the limits for the amplitudes are given in Section \ref{sec:interpretation:model} and all phases are generated randomly according to a flat distribution between $-\pi$ and $\pi$. The masses and widths of the resonances are kept constant at their central values, given in Table \ref{resonances}. 

For each sample, $\kappa$ is computed, as the magnitude of the expression in Equation \ref{kappadefinition}, resulting in a distribution of $\kappa$ values estimated by the model, shown in Figure \ref{kappadistribution}. The mean and standard deviation of the resulting distribution provides an estimate of the central value and uncertainty of $\kappa$,  $0.95 \pm 0.04$. However, it is considered necessary to enhance the uncertainty of this estimate in order to account for the skewness of the distribution, therefore a final value of $\kappa = 0.95 \pm 0.06$ is used in this thesis to extract the physics parameters of interest.

\begin{figure}[h]
\centering
\includegraphics[trim = 0mm 0mm 0mm 8mm, clip, width=0.5\linewidth]{figures/results/kappa.pdf}
\caption{Distribution of the values of $\kappa$ from 1000 samples generated according the amplitude model.}
\label{kappadistribution}
\end{figure}

\section{Four-body phase space acceptance variations}
\label{sec:interpretation:inputs}

Several parameters relating to the \Dz decays, which appear in Eqs.~\ref{exp_Acp} - \ref{exp_Rpm4body}, are required to interpret the results of the \CP observables from the \CP fit into information on $r_B$, $\delta_B$ and \Pgamma. These parameters, namely $r_D^{K\pi}$, $\delta_D^{K\pi}$, $r_D^{K3\pi}$, $\delta_D^{K3\pi}$, $R_{K3\pi}$ and $F_{4\pi}$, are taken from previous measurements and used as inputs alongside the \CP observables and the coherence factor, $\kappa$. For multibody \decay{\Dz}{\Kmp\pipm\pimp\pipm} decays the strong phase varies over the phasespace. In this analysis, interference effects are diluted by averaging the strong phase variation, which is accounted for by the four-body coherence factor, $R_{K3\pi}$~\cite{charmk3pi,LHCb-PAPER-2015-057}. The four-body \Dz decay mode \decay{\Dz}{\pip\pim\pip\pim} is an approximate \CP eigenstate; as it is not a pure \CP eigenstate its sensitivity to \Pgamma is reduced. The parameter $F_{4\pi} \sim 0.75$~\cite{charm4pi} is the \CP-even fraction, which accounts for the dilution effect.

These parameters $F_{4\pi}$ and $R_{K3\pi}$ have been measured at CLEO and \lhcb~\cite{charmk3pi,LHCb-PAPER-2015-057,charm4pi}. However, the interpretation of \lhcb results have a dependence on \lhcb acceptance across the four-body phase space. Therefore, the $F_{4\pi}$ and $R_{K3\pi}$ parameter values taken from CLEO may need to have corrections applied to be used for the interpretation of results from \lhcb. The difference in \lhcb acceptance across the four-body phase space would also affect the efficiency corrections for $A_{\pi\pi\pi\pi}$. These effects are expected to be negligible.

Figures \ref{dalitzk3pi} and \ref{dalitz4pi} show plots of projections of the four-body phase space distributions for \kpipipi and \pipipipi modes respectively. These plots are comparisons of distributions for simulated generator level signal distributions without any acceptance effects, and fully reconstructed simulated signal events used in this analysis. These distributions are very similar, which suggests that any differences due to \lhcb acceptance effects are very small, implying that the use of the \kpipipi coherence factor, $R_{K3\pi}$, and fractional \CP content, $F_{4\pi}$, from CLEO can be used directly in the interpretation of \lhcb results. 

The distributions in Figure \ref{dalitz4pi} are shown to be very similar, which suggests that the fractional \CP content, $F_{4\pi}$, from CLEO can be used directly in the interpretation of \lhcb results. Therefore, the value of $F_{4\pi}$, taken directly from the CLEO measurement, is $0.734 \pm 0.028$~\cite{charm4pi}. There is has been no amplitide model developed for the \decay{\Dz}{\pim\pip\pim\pip} decay, therfore it is not possible to perform detailed quantitative studies into the effect of any acceptance difference on the value of $F_{4\pi}$. When considering the efficiency corrections to $A_{\pi\pi\pi\pi}$, any differences in the distributions in Figure \ref{dalitz4pi} would affect the efficiency by less than 1\%. A correction on the central value of $A_{\pi\pi\pi\pi}$ of $\sim$1\% makes no significant difference to the results given that the statistical and systematic uncertainties are significantly larger than this correction. Therefore, it is concluded that small variations in efficiency lead to a negligible correction in the observables. 

\begin{figure}[h]
\centering
\includegraphics[width=0.9\linewidth]{figures/results/dalitzDist_KPiPiPi.pdf}
\caption{Distributions of invariant mass squared for all two- and three-particle \Dz daughter combinations with simulated generator level events (blue) and fully reconstructed and selected events (red) in the \kpipipi mode. The variable $sXY$ is defined as $(p_X + p_Y)^2$, where $X$ and $Y$ represent labels that refer to different \Dz daughters, and $p_X$ is the four-momentum of particle $X$. The different \Dz daughters labels are defined by: \decay{\Bm}{\D(K^-_1\pi^+_2\pi^-_3\pi^+_4)\Kstarm}.}
\label{dalitzk3pi}
\end{figure}

\begin{figure}[h]
\centering
\includegraphics[width=0.9\linewidth]{figures/results/dalitzDist_PiPiPiPi.pdf}
\caption{Distributions of invariant mass squared for all two- and three-particle \Dz daughter combinations with simulated generator level events (blue) and fully reconstructed and selected events (red) in the \pipipipi mode. The variable $sXY$ is defined as $(p_X + p_Y)^2$, where $X$ and $Y$ represent labels that refer to different \Dz daughters, and $p_X$ is the four-momentum of particle $X$. The different \Dz daughters labels are defined by: \decay{\Bm}{\D(\pi^-_1\pi^+_2\pi^-_3\pi^+_4)\Kstarm}.}
\label{dalitz4pi}
\end{figure}

Further studies are performed to assess whether the parameters $R_{K3\pi}$ and $\delta_D^{K3\pi}$ require any corrections, when used in the interpretation, due to variations in acceptance over the four-body phase space. To assess the effects of small variation in the four-body phase space for \kpipipi decays, the coherence factor and strong phase are calculated from \kpipipi amplitude models, both under the assumption of uniform acceptance and total \lhcb acceptance. The differences between these two scenarios gives an estimate of the corrections that should be applied to the CLEO result before it is used in the \lhcb interpretation. The difference is calculated to be 0.002 for the coherence factor and 0.7$^{\circ}$ for the strong phase difference. The values of the inputs taken from CLEO/LHCb are $R_{K3\pi} = 0.43^{+0.17}_{-0.13}$ and $\delta_D^{K3\pi} = \left(128^{+28}_{-17}\right)$~\cite{charmk3pi}. Therefore, the size of the corrections due to the \lhcb phase space acceptance are negligible in comparison to the CLEO uncertainties and hence no further action is taken.


\section{Results in terms of $r_B$, $\delta_B$ and \Pgamma}
\label{sec:interpretation:gammadini}

The \CP observables, measured in the \CP fit and listed in Section \ref{sec:cpfit:summary}, can be used to gain information on the physics parameters of interest: $r_B$, $\delta_B$ and \Pgamma. These observables can be related to the physics parameters via Equations ~\ref{exp_Acp} - ~\ref{exp_Rpm4body}. The coherence factor is estimated, as described in the Section \ref{sec:interpretation:coherence}, and used as an input in the extraction of information on $r_B$, $\delta_B$ and \Pgamma. The parameters $r_D^{K\pi}$, $\delta_D^{K\pi}$, $r_D^{K3\pi}$, $\delta_D^{K3\pi}$, $R_{K3\pi}$ and $F_{4\pi}$ are required as external inputs and are taken from Ref.~\cite{HFAG,charmk3pi,LHCb-PAPER-2015-057,charm4pi}. The \CP observables and other parameters are combined, with the parameters $\kappa$, $r_D^{K\pi}$, $\delta_D^{K\pi}$, $r_D^{K3\pi}$, $\delta_D^{K3\pi}$, $R_{K3\pi}$ and $F_{4\pi}$ constrained to the values given in Table \ref{inputparameters}. 

\begin{table}
\centering
\begin{tabular}{cc}
Fit parameter & Value \\
\hline
$\kappa$ & $0.96 \pm 0.06$ \\
$r_D^{K\pi}$ & $0.0591 \pm 0.0003$ \\
$\delta_D^{K\pi}$ & $191.8 \pm 12.1$ \\
$r_D^{K3\pi}$ & $0.0549 \pm 0.0006$ \\
$\delta_D^{K3\pi}$ & $128 \pm 28$ \\
$R_{K3\pi}$ & $0.43 \pm 0.17$ \\
$F_{CP}$ & $0.737 \pm 0.028$
\end{tabular}
\caption{Values of the external inputs used as constraints. These values are taken from Ref.~\cite{HFAG,charmk3pi,LHCb-PAPER-2015-057,charm4pi}.}
\label{inputparameters}
\end{table}

Constainting these parameters, as opposed to using them as fixed inputs, allows the possibility of the \CP observables providing additional information on the values of these parameters. A global $\chi^2$ minimisation is performed, taking correlations into account, where
\begin{equation}
\chi^2 = \sum_i \chi^2_i = (x(\theta) - x_0)^TV_0^{-1}(x(\theta)-x_0) \text{ . }
\end{equation}
In this expression, $x(\theta)$ is the set of observables calculated from the fundamental set of physics parameters, $\theta$. The set, $x_0$, are the measured values of the observables and $V_0$ is the covariance matrix. The sensitivity of a given parameter is determined by calculating the $\chi^2$ at fixed points in parameter space. The difference between this calculated $\chi^2$ value and that of the global minimum, $\Delta\chi^2$, quantifies the confidence in the global minimum. 

Using the measured values of the \CP observables, their uncertainties and the covariance matrices, a global $\chi^2$ minimisation is performed, resulting in a minimum $\chi^2$ of 3.0 with 9 degrees of freedom. A scan of physics parameters is performed for a range of values and the difference in $\chi^2$ between the parameter scan values and the global minimum, $\Delta\chi^2$, is evaluated. The confidence level for any pair of parameters is calculated assuming that these are normally distributed, which enables the $\Delta \chi^2 = 2.30,\ 6.18,\ 11.8$ contours to be drawn, corresponding to 68.3\%, 95.5\%, 99.7\% confidence levels, respectively. 

Figures \ref{gammadiniplots2body} and \ref{gammadiniplotsallmodes} give 2D contour plots of $r_B$ versus \Pgamma and $\delta_B$ versus \Pgamma. Figure \ref{gammadiniplots2body} shows the contour plots using the \CP observables from the two-body modes only and Figure \ref{gammadiniplotsallmodes} show the contour plots using the \CP observables from both the two- and four-body decays. The addition of the four-body modes improves the constraints on the physics parameters and provides additional distinction between the two minima. The results from this $\chi^2$ minimisation procedure of the values of the external inputs, listed in Table \ref{inputparameters}, remain effectively unchanged, hence there is little sensitivity to these parameters with the exisiting data.

The data are consistent with the value of $\gamma$ indicated by previous measurements~\cite{LHCb-PAPER-2016-032, CKMFitter}, $\sim 70^\circ$. The values of $r_B$, $\delta_B$ and \Pgamma are determined at the point where the global $\chi^2$ of the fit is minimised. The values of these physics parameters are calculated to be:
\begin{align*}
r_B &= 0.113^{+0.015}_{-0.021} \\
\delta_B &= \left(43.3^{+19.0}_{-19.0}\right)^{\circ} \\
\Pgamma &= \left(40.0^{+18.0}_{-16.0}\right)^{\circ} 
\end{align*}
The confidence level distributions for $r_B$, $\delta_B$ and \Pgamma are not Gaussian beyond 1$\sigma$, and $\delta_B$ and \Pgamma contain a second minimum, as can be seen from Figure \ref{gammadiniplotsallmodes}. Therefore, the 1$\sigma$ intervals quoted cannot be extrapolated to consider 2$\sigma$ or 3$\sigma$ values. 

\begin{figure}[h]
\centering
\subfloat[$r_B$ versus \Pgamma]{\includegraphics[width=0.5\linewidth]{figures/interpretation/rBu_dkstar_gamma_2Dscan_nomixing_2body.pdf}}
\subfloat[$\delta_B$ versus \Pgamma]{\includegraphics[width=0.5\linewidth]{figures/interpretation/deltaBu_dkstar_gamma_2Dscan_nomixing_2body.pdf}}
\caption{Contour plots showing 2D scans of the physics parameters using the two-body modes only. The dashed lines represent the $\Delta \chi^2 = 2.30,\ 6.18,\ 11.8$ contours, corresponding to 68.3\%, 95.5\%, 99.7\% confidence levels (CL), respectively. The colour scale represents 1 - CL.}
\label{gammadiniplots2body}
\end{figure}

\begin{figure}[h]
\centering
\subfloat[$r_B$ versus \Pgamma]{\includegraphics[width=0.5\linewidth]{figures/interpretation/rBu_dkstar_gamma_2Dscan_nomixing_all.pdf}}
\subfloat[$\delta_B$ versus \Pgamma]{\includegraphics[width=0.5\linewidth]{figures/interpretation/deltaBu_dkstar_gamma_2Dscan_nomixing_all.pdf}}
\caption{Contour plots showing 2D scans of the physics parameters using both the two- and four-body modes. The dashed lines represent the $\Delta \chi^2 = 2.30,\ 6.18,\ 11.8$ contours, corresponding to 68.3\%, 95.5\%, 99.7\% confidence levels (CL), respectively. The colour scale represents 1 - CL.}
\label{gammadiniplotsallmodes}
\end{figure}

%\begin{table}
%\centering
%\begin{tabular}{cccc}
%Fit parameter & Value & Negative error & Positive error \\
%\hline
%\Pgamma & 41 & -16 & 19 \\
%$r_B$ & 0.113 & -0.019 & 0.016 \\
%$\delta_B$ & 43 & -20 & 19 \\
%$\kappa$ & 0.96 & -0.06 & 0.06 \\
%$r_D^{K\pi}$ & 0.0591 & -0.0003 & 0.0003 \\
%$\delta_D^{K\pi}$ & 193 & -11 & 11 \\
%$r_D^{K3\pi}$ & 0.0549 & -0.0006 & 0.0006 \\
%$\delta_D^{K3\pi}$ & 134 & -22 & 22 \\
%$R_{K3\pi}$ & 0.44 & -0.15 & 0.15 \\
%$F_{CP}$ & 0.74 & -0.03 & 0.03
%\end{tabular}
%\caption{Results of fit parameters.}
%\label{gammadinifit}
%\end{table}

\section{Expected future sensitivity to $r_B$, $\delta_B$ and \Pgamma}
\label{sec:interpretation:futuresensitivity}

In this thesis, the \btodkst mode has shown an increase of three times the yield per unit integrated luminosity from \runone to \runtwo, mainly due to the increase in centre of mass energy of the $pp$ collisions from 7 and 8\tev to 13\tev. As previously stated, the data used in this analysis was collected in 2011 and 2012, forming the \runone \dataset, and 2015 and 2016, forming part of the \runtwo \dataset. The current running period of the LHC, namely \runtwo, is currently ongoing until the end of 2018. The next running period, Run 3, will take place between 2021 and 2023.

The projected yields for the end of Run 2 and the end of Run 3 are estimated using the forecasts for the stored integrated luminosity and expected improvements in detector performance during these data-taking periods. The projections of the integrated luminosities can be found in \tab~\ref{projectedyields}. In order to do make projections about the future sensitivity of the \btodkst mode, it is assumed that the yield per unit integrated luminosity remains the same for the rest of \runtwo, which is a reasonable assumption as there are not expected to be any significant changes to the running conditions. 

However, between Run 2 and Run 3 the detector will undergo an upgrade to improve detector performance. Improvements to the detector, trigger and reconstruction will be made, including moving to a fully software-based trigger~\cite{CERN-LHCC-2014-016} and the rebuilding of many detector components. These improvements to the detector allow the experiment to run at an instantaneous luminosity of $2 \times 10^{33} cm^{-2}s^{-1}$, five times larger than the current conditions~\cite{CERN-LHCC-2014-016}. Therefore, Run 3 is expected to produce a minimum of 7\invfb per year for two years. However, the upgrade not only increases the integrated luminosity collected, but also increases the yield obtained per unit of intergrated luminosity. The upgrade to a fully software-based trigger improves the trigger efficiencies for the fully hadronic modes by about a factor of two compared to \runone. Additionally, the centre of mass energy is expected to increase to 14\tev. Due to the observed improvement in \runtwo due to the increased centre of mass energy and the expected improvements in Run 3 from the upgrade of the detector, yields of \B meson decay in Run 3 are expected to increase by a factor of $\sim$17.6 compared to \runone, assuming no improvements in the analysis procedure. The result of these assumptions are shown in the projected estimates in Table~\ref{projectedyields}.

\begin{table}
\resizebox{\textwidth}{!}{
\begin{tabular}{cccc}
\hline
Year & Integrated Luminosity & \kpi yield & Yield per \invfb \\
\hline
\runone & 3\invfb & 725 & 242 \\
\runtwo (up to 2016) & 1.8\invfb & 1390 & 771 \\
\textbf{Run 2 (after 2016)} & \textbf{3.2\invfb} & \textbf{2466} & \textbf{771} \\
\textbf{Run 3} & \textbf{14\invfb} & \textbf{23240} & \textbf{1660} \\
\hline
\end{tabular}}
\caption{Yields and projected yields for different data-taking periods of the LHC. The entries in bold are projected yields, whereas the other entries refer to data used in this thesis. Projected results are justified in the text, with information taken from Ref.~\cite{CERN-LHCC-2014-016}.}
\label{projectedyields}
\end{table}

% Luminosity/Yield factor for Run 2: 2.2
% Luminosity/Yield factor for Run 3: 7.3

Using the assumptions made in \tab~\ref{projectedyields}, the projected future sensitivity to the physics parameters \rb, \deltab and \Pgamma are estimated. Figure \ref{gammadiniplotsrun2} gives the projected results at the end of Run 2 as 2D contour plots of \rb versus \Pgamma and \deltab versus \Pgamma. The equivalent plots are shown in Figure \ref{gammadiniplotsrun3} for the end of Run 3. 

\begin{figure}[h]
\centering
\subfloat[\rb versus \Pgamma]{\includegraphics[width=0.5\linewidth]{figures/interpretation/rBu_dkstar_gamma_2Dscan_nomixing_Run2.pdf}}
\subfloat[\deltab versus \Pgamma]{\includegraphics[width=0.5\linewidth]{figures/interpretation/deltaBu_dkstar_gamma_2Dscan_nomixing_Run2.pdf}}
\caption{Contour plots showing projected 2D scans of the physics parameters at the end of \runtwo, assuming the central values remain the same. The dashed lines represent the $\Delta \chi^2 = 2.30,\ 6.18,\ 11.8$ contours, corresponding to 68.3\%, 95.5\%, 99.7\% confidence levels (CL), respectively. The colour scale represents 1 - CL.}
\label{gammadiniplotsrun2}
\end{figure}

\begin{figure}[h]
\centering
\subfloat[\rb versus \Pgamma]{\includegraphics[width=0.5\linewidth]{figures/interpretation/rBu_dkstar_gamma_2Dscan_nomixing_Run3.pdf}}
\subfloat[\deltab versus \Pgamma]{\includegraphics[width=0.5\linewidth]{figures/interpretation/deltaBu_dkstar_gamma_2Dscan_nomixing_Run3.pdf}}
\caption{Contour plots showing projected 2D scans of the physics parameters at the end of Run 3, assuming the central values remain the same. The dashed lines represent the $\Delta \chi^2 = 2.30,\ 6.18,\ 11.8$ contours, corresponding to 68.3\%, 95.5\%, 99.7\% confidence levels (CL), respectively. The colour scale represents 1 - CL.}
\label{gammadiniplotsrun3}
\end{figure}

It can be seen that the sensitivity continues to improve with more data, because the uncertainties are dominated by statistical uncertainty. For the \decay{\Bm}{\D\Km} channel, the statisitical uncertainties for some of the \CP observables are already of the same order as the sysytematic uncertainties. Therefore, the sensitivity to \Pgamma will not continue to decrease at the same rate when the size of the dataset increases. However, this is not the case for the \btodkst channel, due to the larger statistical uncertainties. At the end of Run 3 the uncertainty on \Pgamma has reduced to about $8^{\circ}$. Uncertainties in the measurements of the hadronic parameters \rb and \deltab will also reduce signficantly. The \btodkst channel will continue to be statisitcally dominated, and therefore will continue to benefit from the increased dataset in Run 3 and beyond. This channel will be increasingly helpful in the future in providing additional constraints when combining a range of \Pgamma-sensitive analyses to improve on the current world's best measurement of \Pgamma.

\clearpage
