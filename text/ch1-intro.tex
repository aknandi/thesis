%\begin{savequote}[8cm]
%\textlatin{Neque porro quisquam est qui dolorem ipsum quia dolor sit amet, consectetur, adipisci velit...}
%
%There is no one who loves pain itself, who seeks after it and wants to have it, simply because it is pain...
%  \qauthor{--- Cicero's \textit{de Finibus Bonorum et Malorum}}
%\end{savequote}

%The only true wisdom is knowing you know nothing - Socrates
%I'd rather have questions that I cannot answer than answers that I cannot question - Richard Feynmann
%Research is what I am doing when I don't know what I'm doing - Wernher von Braun
%Science is a way of thinking much more than it is a body of knowledge - Carl Sagan
%Two things are infinite: the universe and human stupidity; and I'm not sure about the universe - Albert Einstein
%There's nothing quite as frightening as someone who knows they are right - Michael Faraday

\chapter{\label{ch:1-intro}Introduction} 

The field of particle physics has arisen from curiosity and desire to understand the world around us. Throughout the last century a greater understanding of the most fundamental particles that constitute matter and the interactions between them has been developed, which has culminated in the Standard Model (SM) of particle physics. This elegant and well tested model represents our current best understanding of the Universe.

\CP violation enters the SM through a single phase in the CKM quark mixing matrix. The unitarity constraints of this matrix can be graphically represented by a triangle in the complex plane. One of the angles in this triangle, conventionally labelled \Pgamma, is a very useful parameterisation of \CP violation, due to the fact that it can be measured in purely tree-level processes. While processes that contain loops are expected to have sensitivity to physics beyond the SM, a tree-level measurement acts as a SM ``benchmark''. Therefore, by comparing tree-level SM measurements of \Pgamma to indirect, loop-level measurements, any deviation observed indicates new physics. Hence the uncertainties on direct measurements of \Pgamma must be minimised in order to make such comparisons meaningful.

This thesis investigates \CP violation in \btodkst decays, where \D is a superposition of \Dz and \Dzb decaying to two- and four-body final states containing charged pions and kaons. The sensitivity of these decays to the CKM angle \Pgamma is studied. The measurement involves isolating the \btodkst families of decays before measuring various observables relating to the yields, which are sensitive to \CP violation. Information on the CKM angle \Pgamma is extracted, along with other hadronic parameters of the decay.

The data used in this thesis are collected by the LHCb detector, located at the CERN Large Hadron Collider (LHC). The experiment specialises in \CP violation measurements of \bquark and \cquark hadrons. In total the data used in this thesis consists of 1\invfb and 2\invfb of $pp$ collisions at $\sqrt{s} = 7\tev \text{ and } 8\tev$ collected in 2011 and 2012 respectively, and 1.8\invfb at $\sqrt{s} = 13\tev$ collected in 2015 and 2016.

Chapter \ref{ch:2-background} covers the background theory behind \CP violation in the Standard Model. The unitarity triangle and CKM angle \Pgamma are discussed, followed by the methods to extract \CP violation measurements from \btodkst decays. Previous \Pgamma measurements using these decays are also summarised. 

In Chapter \ref{ch:3-detector} an overview of the \lhcb detector is presented, covering each of the sub-detectors: \velo, tracking systems, dipole magnet, \rich sub-detectors, electromagnetic and hadronic calorimeters and muon system. This chapter also introduces the trigger system and reconstruction, as well as discussing the process creating simulated event samples in the \lhcb detector.

The original research performed by the author is primarily documented
in Chapters \ref{ch:4-selection} - \ref{ch:6-interpretation}. Chapter \ref{ch:4-selection} presents a discussion of the selection requirements of the analysis including, reconstruction, particle identification, reducing so-called peaking backgrounds and the multivariate analysis techniques used. This chapter also presents the mass parameterisation of the favoured modes, covering the shapes for each of the model components, before revealing the full mass fits.

Chapter \ref{ch:5-cpfit} covers the simultaneous fit that is used to extract the \CP observables. The setup of the simultaneous fit is discussed in detail, before moving onto the results obtained from the fit and the systematic uncertainties. Finally, the results are interpreted in terms of the parameters of interest, \rb, \deltab and \Pgamma, in Chapter \ref{ch:6-interpretation}, where \rb and \deltab are the hadronic parameters of the \Bm-meson decay. Here the external inputs used in the interpretation are discussed, before providing results on the sensitivity to \rb, \deltab and \Pgamma in the \btodkst decay channel. Additionally, the expected future sensitivity to these parameters in \btodkst decays is discussed.

The work presented in this thesis was published in a single article in the Journal of High Energy Physics \textbf{11} (2017) 156. 




\minitoc


