% $Id: introduction.tex 87303 2016-02-08 13:44:29Z lafferty $
\begin{savequote}[8cm]
\textlatin{Neque porro quisquam est qui dolorem ipsum quia dolor sit amet, consectetur, adipisci velit...}

There is no one who loves pain itself, who seeks after it and wants to have it, simply because it is pain...
  \qauthor{--- Cicero's \textit{de Finibus Bonorum et Malorum}}
\end{savequote}

\chapter{\label{ch:2-background}\CP violation and measurements of CKM angle \Pgamma} 

\minitoc

\section{\CP violation}

\section{The CKM matrix}

\section{The CKM unitarity triangle}

\section{Tree-level determination of \Pgamma using \decay{\Bpm}{\D\Kstarpm} decays}

\subsection{The GLW method}

\subsection{The ADS method}

\subsection{Physical observables}

Twelve quantities, collectively referred to as \CP observables, are measured in this analysis:

\begin{itemize}
\item{The \CP asymmetry for the favoured decay mode
{\footnotesize
\begin{equation}
A_{\kaon\pi} = \frac{\Gamma\left(\decay{\Bm}{\D(\Km\pip)\Kstarm}\right) - \Gamma\left(\decay{\Bp}{\D(\Kp\pim)\Kstarp}\right)}{\Gamma\left(\decay{\Bm}{\D(\Km\pip)\Kstarm}\right) + \Gamma\left(\decay{\Bp}{\D(\Kp\pim)\Kstarp}\right)} \text{ .}
\label{eqn:Akpi}
\end{equation}}%
\noindent%
This asymmetry should be essentially zero due to the very small interference expected in this configuration of \B- and \D-meson decays.
}
\item{The \CP asymmetry for the \decay{\D}{\Kp\Km} decay mode
{\footnotesize
\begin{equation}
A_{\kaon\kaon} = \frac{\Gamma\left(\decay{\Bm}{\D(\Kp\Km)\Kstarm}\right) - \Gamma\left(\decay{\Bp}{\D(\Kp\Km)\Kstarp}\right)}{\Gamma\left(\decay{\Bm}{\D(\Kp\Km)\Kstarm}\right) + \Gamma\left(\decay{\Bp}{\D(\Kp\Km)\Kstarp}\right)} \text{ . }
\label{eqn:Akk}
\end{equation}
}}
\item{The \CP asymmetry for the \decay{\D}{\pip\pim} decay mode
{\footnotesize
\begin{equation}
A_{\pi\pi} = \frac{\Gamma\left(\decay{\Bm}{\D(\pip\pim)\Kstarm}\right) - \Gamma\left(\decay{\Bp}{\D(\pip\pim)\Kstarp}\right)}{\Gamma\left(\decay{\Bm}{\D(\pip\pim)\Kstarm}\right) + \Gamma\left(\decay{\Bp}{\D(\pip\pim)\Kstarp}\right)} \text{ . }
\label{eqn:Apipi}
\end{equation}}
}
Assuming negligible direct \CP violation in \D-meson decays, the observables $A_{\kaon\kaon}$ and $A_{\pi\pi}$ should be equal and are often labelled together as $A_{CP+}$.
\item{The ratio of the rate for the \decay{\D}{\Kp\Km} decay mode to that of the favoured decay mode, scaled by the branching fractions
{\footnotesize
\begin{equation}
R_{\kaon\kaon} = \frac{\Gamma\left(\decay{\Bm}{\D(\Kp\Km)\Kstarm}\right) + \Gamma\left(\decay{\Bp}{\D(\Kp\Km)\Kstarp}\right)}{\Gamma\left(\decay{\Bm}{\D(\Km\pip)\Kstarm}\right) + \Gamma\left(\decay{\Bp}{\D(\Kp\pim)\Kstarp}\right)} \cdot \frac{\BR(D^0 \to K^-\pi^+)}{\BR(D^0 \to K^+K^-)} \text{ . }
\label{eqn:Rkk}
\end{equation}
}}
\item{The ratio of the rate for the \decay{\D}{\pip\pim} decay mode to that of the favoured decay mode, scaled by the branching fractions
{\footnotesize
\begin{equation}
R_{\pi\pi} = \frac{\Gamma\left(\decay{\Bm}{\D(\pip\pim)\Kstarm}\right) + \Gamma\left(\decay{\Bp}{\D(\pip\pim)\Kstarp}\right)}{\Gamma\left(\decay{\Bm}{\D(\Km\pip)\Kstarm}\right) + \Gamma\left(\decay{\Bp}{\D(\Kp\pim)\Kstarp}\right)} \cdot \frac{\BR(D^0 \to K^-\pi^+)}{\BR(D^0 \to \pi^+\pi^-)} \text{ . }
\label{eqn:Rpipi}
\end{equation}}
}
Assuming negligible direct \CP violation in \D-meson decays, the observables $R_{\kaon\kaon}$ and $R_{\pi\pi}$ should be equal and are often labelled together as $R_{CP+}$.
\item{The ratio of the rate for the ADS decay mode to that of the favoured decay mode for \Bp decays
{\footnotesize
\begin{equation}
R^+_{K\pi} = \frac{\Gamma\left(\decay{\Bp}{\D(\Km\pip)\Kstarp}\right)}{\Gamma\left(\decay{\Bp}{\D(\Kp\pim)\Kstarp}\right)} \text{ . }
\label{eqn:Rplus}
\end{equation}
}}
\item{The ratio of the rate for the ADS decay mode to that of the favoured decay mode for \Bm decays
{\footnotesize
\begin{equation}
R^-_{K\pi} = \frac{\Gamma\left(\decay{\Bm}{\D(\Kp\pim)\Kstarm}\right)}{\Gamma\left(\decay{\Bm}{\D(\Km\pip)\Kstarm}\right)} \text{ . }
\label{eqn:Rminus}
\end{equation}
}}
\item{The \CP asymmetry for the favoured \decay{\Dz}{\Km\pip\pim\pip} decay mode
{\footnotesize
\begin{equation}
A_{\kaon\pi\pi\pi} = \frac{\Gamma\left(\decay{\Bm}{\D(\Km\pip\pim\pip)\Kstarm}\right) - \Gamma\left(\decay{\Bp}{\D(\Kp\pim\pip\pim)\Kstarp}\right)}{\Gamma\left(\decay{\Bm}{\D(\Km\pip\pim\pip)\Kstarm}\right) + \Gamma\left(\decay{\Bp}{\D(\Kp\pim\pip\pim)\Kstarp}\right)} \text{ . }
\label{eqn:Akpipipi}
\end{equation}}%
\noindent%
This asymmetry should be essentially zero due to the very small interference expected in this configuration of \B- and \D-meson decays.
}
\item{The \CP asymmetry for the \decay{\D}{\pip\pim\pip\pim} decay mode
{\footnotesize
\begin{equation}
A_{\pi\pi\pi\pi} = \frac{\Gamma\left(\decay{\Bm}{\D(\pip\pim\pip\pim)\Kstarm}\right) - \Gamma\left(\decay{\Bp}{\D(\pip\pim\pip\pim)\Kstarp}\right)}{\Gamma\left(\decay{\Bm}{\D(\pip\pim\pip\pim)\Kstarm}\right) + \Gamma\left(\decay{\Bp}{\D(\pip\pim\pip\pim)\Kstarp}\right)} \text{ . }
\label{eqn:Apipipipi}
\end{equation}
}}
\item{The ratio of the rate for the \decay{\D}{\pip\pim\pip\pim} decay mode to that of the favoured decay mode, scaled by the branching fractions
{\footnotesize
\begin{equation}
R_{\pi\pi\pi\pi} = \frac{\Gamma\left(\decay{\Bm}{\D(\pip\pim\pip\pim)\Kstarm}\right) + \Gamma\left(\decay{\Bp}{\D(\pip\pim\pip\pim)\Kstarp}\right)}{\Gamma\left(\decay{\Bm}{\D(\Km\pip\pim\pip)\Kstarm}\right) + \Gamma\left(\decay{\Bp}{\D(\Kp\pim\pip\pim)\Kstarp}\right)} \cdot \frac{\mathcal{B}(D^0 \to \Km\pip\pim\pip)}{\mathcal{B}(D^0 \to \pip\pim\pip\pim)} \text{ . }
\label{eqn:Rpipipipi}
\end{equation}}
}
\item{The ratio of the rate for the four-body ADS decay mode to that of the four-body favoured decay mode for \Bp decays
{\footnotesize
\begin{equation}
R^+_{K\pi\pi\pi} = \frac{\Gamma\left(\decay{\Bp}{\D(\Km\pip\pim\pip)\Kstarp}\right)}{\Gamma\left(\decay{\Bp}{\D(\Kp\pim\pip\pim)\Kstarp}\right)} \text{ . }
\label{eqn:Rplus4body}
\end{equation}
}}
\item{The ratio of the rate of the four-body ADS decay mode to that of the four-body favoured decay mode for \Bm decays
{\footnotesize
\begin{equation}
R^-_{K\pi\pi\pi} = \frac{\Gamma\left(\decay{\Bm}{\D(\Kp\pim\pip\pim)\Kstarm}\right)}{\Gamma\left(\decay{\Bm}{\D(\Km\pip\pim\pip)\Kstarm}\right)} \text{ . }
\label{eqn:Rminus4body}
\end{equation}
}}
\end{itemize}

\noindent
In contrast to the GLW decay modes, for the ADS decay mode the ratios are measured separately for the positive and negative charges. These observables have statistical uncertainties that are more Gaussian than asymmetry variables for the very low yields that are expected. 

Several \CP observables measured in this analysis can be related to the physics parameters to be determined, namely \Pgamma, $r_B$, $\delta_B$ and $\kappa$. The parameter $r_B$ is the magnitude of the ratio between the suppressed and favoured amplitudes of the \B-meson decay and $\delta_B$ is the strong-phase difference between these amplitudes. The expectation value is $r_B$ $\sim 0.1$, similar to that in the \decay{\Bm}{\D\Km} decay due to the similarity of the hadronic interactions. Both $r_B$ and $\delta_B$ are averaged over the region of \D\KS\pim phase space corresponding to the \Kstarm-meson selection window. The coherence factor $\kappa$ accounts for the contribution of \CP violation from $\Bm \to D \KS \pim$ decays that are not due to an intermediate $K^*(892)^{-}$ resonance~\cite{Gronau2003198}, where $\kappa = 1$ denotes pure $K^*(892)^{-}$. Assuming a negligible effect from charm mixing~\cite{charmmixing}, the relationships between the \CP observables and physics parameters are given in the following equations,

\begin{equation}
A_{\CP+} = \frac{2 \kappa r_B\sin\delta_B\sin\gamma}{1 + r_B^2 + 2 \kappa r_B\cos\delta_B\cos\gamma}
\label{exp_Acp}
\end{equation}

\begin{equation}
R_{\CP+} = 1 + r_B^2 + 2 \kappa r_B\cos\delta_B\cos\gamma
\label{exp_Rcp}
\end{equation}

\begin{equation}
R^{\pm}_{K\pi} = \frac{r_B^2 + \left(r_D^{K\pi}\right)^2 + 2\kappa r_B r_D^{K\pi} \cos(\delta_B + \delta_D^{K\pi} \pm \gamma)}{1 + r_B^2\left(r_D^{K\pi}\right)^2 + 2\kappa r_B r_D^{K\pi} \cos(\delta_B - \delta_D^{K\pi} \pm \gamma)}
\label{exp_Rpm}
\end{equation}

\begin{equation}
A_{\pi\pi\pi\pi} = \frac{2 \kappa\left(2F_{4\pi} - 1\right) r_B\sin\delta_B\sin\gamma}{1 + r_B^2 + 2 \kappa\kappa_{4\pi} r_B\cos\delta_B\cos\gamma}
\label{exp_A4pi}
\end{equation}

\begin{equation}
R_{\pi\pi\pi\pi} = 1 + r_B^2 + 2 \kappa\left(2F_{4\pi} - 1\right) r_B\cos\delta_B\cos\gamma
\label{exp_R4pi}
\end{equation}

\begin{equation}
R^{\pm}_{K\pi\pi\pi} = \frac{r_B^2 + \left(r_D^{K3\pi}\right)^2 + 2\kappa r_B \kappa_{K3\pi} r_D^{K3\pi} \cos(\delta_B + \delta_D^{K3\pi} \pm \gamma)}{1 + \left(r_Br_D^{K3\pi}\right)^2 + 2\kappa r_B \kappa_{K3\pi} r_D^{K3\pi} \cos(\delta_B - \delta_D^{K3\pi} \pm \gamma)} \text{ .}
\label{exp_Rpm4body}
\end{equation}
These relationships depend on several parameters describing the \D-meson decay, which are taken from other existing measurements. The parameters $r_D^{K\pi}$ and $\delta_D^{K\pi}$ are the magnitude of the amplitude ratio and the strong phase difference between the suppressed and favoured amplitudes of the \D-meson decay, namely \decay{\Dz}{\Kp\pim} and \decay{\Dz}{\Km\pip} respectively. Similarly, the parameters $r_D^{K3\pi}$ and $\delta_D^{K3\pi}$ are the equivalent quantities for the decays \decay{\Dz}{\Kp\pim\pip\pim} and \decay{\Dz}{\Km\pip\pim\pip}, averaged over phase space. Two-body \decay{\D}{\Kmp\pipm} decays are characterised by a single strong phase, however for multibody \decay{\D}{\Kmp\pipm\pimp\pipm} decays the strong phase varies over the phase space. By averaging the strong phase variation the interference effects are diluted, which is accounted for by the parameter $\kappa_{K3\pi}$. The parameter $F_{4\pi} \sim 0.75$~\cite{charm4pi} accounts for the fact that \decay{\D}{\pip\pim\pip\pim}, though predominantly CP even, is not a pure CP-eigenstate.



\section{Previous \Pgamma measurements with \decay{\Bpm}{\D\Kstarpm} decays}

A key characteristic of the Standard Model is that \CP violation originates from a single phase in the CKM quark-mixing matrix~\cite{Cabibbo,KM}. In the Standard Model the CKM matrix is unitary, leading to the condition, $\Vud\Vubs + \Vcd\Vcbs + \Vtd\Vtbs = 0$. This relation is represented as a triangle in the complex plane, with angles $\alpha$, $\beta$ and \Pgamma, and an area proportional to the amount of \CP violation in the Standard Model~\cite{CKMtriangle}. Overconstraining this unitarity triangle can lead to signs of physics beyond the Standard Model. The CKM angle $\Pgamma \equiv \arg\left(-\frac{\Vud{\Vub}^*}{\Vcd{\Vcb}^*}\right)$ is the least well-known angle of the CKM unitarity triangle. The latest \lhcb combination from direct measurements with charged and neutral \B-meson decays to a variety of \D-meson final states is $\Pgamma = \left(72.2^{+6.8}_{-7.3}\right)^{\circ}$~\cite{LHCB-PAPER-2016-032}. Global fits to the CKM triangle from CKMfitter~\cite{CKMfitter} obtain a \Pgamma measurement of $(66.9^{+0.9}_{-3.4})^{\circ}$, where this determination of \Pgamma excludes all direct measurements. The uncertainties on the indirect measurement are dominated by lattice QCD calculations and are expected to decrease as lattice calculations become more accurate. Therefore, precision at the level of $1^\circ$ on a direct measurement of \Pgamma would check the consistency of these two measurements, which is an excellent test of the Standard Model. This can be achieved through a combination of measurements of various \B-meson decays that are sensitive to \Pgamma.

Direct measurements of \Pgamma can be made by exploiting the interference between the \decay{\bquark}{\cquark\uquarkbar\squark} and \decay{\bquark}{\uquark\cquarkbar\squark} transitions. These transitions are present in $\B \to \D^{(*)}\kaon^{(*)}$ decays. This analysis measures \CP violation in $\B^\pm \to \D \Kstar(892)^\pm$ decays, where \D denotes a superposition of \Dz and \Dzb meson states. The effect of the interference is observed by reconstructing the \D meson in a final state accessible to both \Dz and \Dzb meson states, which gives sensitivity to the weak phase \Pgamma. For this analysis, only \D mesons decaying to two or four charged kaons and/or pions are considered. The branching fraction of $\B^- \to \D \Kstar(892)^-$ is of a similar magnitude to \decay{\Bm}{\D\Km},\footnote{The inclusion of charge-conjugate processes is implied, except when discussing ratios or asymmetries between \Bp and \Bm decays.} which has been extensively analysed at \lhcb~\cite{LHCB-PAPER-2016-003, LHCB-PAPER-2014-041, LHCB-PAPER-2015-014}. However, the reconstruction efficiencies associated with the \Kstarm-meson decay, detected through the \KS\pim final state, are expected to be lower due to the presence of a long-lived neutral particle. %The \Kstarm-meson decay to \Km\piz is not considered, as this would have an even lower reconstruction efficiency

Two main classes of \D-meson decays are used. The first employs \D-meson decays into the \CP eigenstates \Kp\Km and \pip\pim; these are referred to here as the ``GLW'' decay modes~\cite{GL,GW}. The second class of decay modes involves \D-meson decays to \Kmp\pipm. One of these is the favoured combination, which is used as a control mode for many aspects of the analysis. In the favoured decay, the pion from the \D meson and the pion from the \Kstarm meson have opposite charge, while in the suppressed decay (referred to here as the ``ADS'' ~\cite{ADS,ADS-2001} decay mode) the pion from the \D meson and pion from the \Kstarm meson have the same charge. The ADS decay mode is a combination of a CKM-favoured \B-meson decay, \decay{\Bm}{\Dz\Kstarm}, followed by a doubly Cabibbo-suppressed \D-meson decay, \decay{\Dz}{\Kp\pim}, and a CKM- and colour-suppressed \B-meson decay, \decay{\Bm}{\Dzb\Kstarm}, followed by a Cabibbo-favoured \D-meson decay, \decay{\Dzb}{\Kp\pim}. Both paths to the same final state have amplitudes of similar size, and interference effects are therefore magnified in comparison to the GLW decay modes, where the decay path via the \Dz meson dominates. Studies of these \D-meson decay modes have been published for \decay{\Bm}{D\Km} and \decay{\Bz}{\D\Kstarz} decays by the \lhcb collaboration~\cite{LHCB-PAPER-2016-003, LHCB-PAPER-2014-028}. The four-body final states \decay{\D}{\Kmp\pipm\pimp\pipm} and \decay{\D}{\pip\pim\pip\pim}, which are included in Ref.~\cite{LHCB-PAPER-2016-003}, are also investigated in this paper. This is the first time four-body \D-meson decay modes have been studied using \decay{\Bm}{\D\Kstarm} decays. The GLW and ADS methods can be extended to these inclusive four-body final states, provided external information is available on the overall behaviour of the intermediate resonances, averaged over phase space~\cite{charminfo,charm4pi}. The \decay{\Bm}{\D\Kstarm} channel has previously been investigated by the BaBar collaboration using a variety of \D-meson  decay modes: \Km\pip, which is not a \CP eigenstate, \CP-even eigenstates \Kp\Km, \pip\pim, and \CP-odd eigenstates \KS\piz, $\KS\phi$ and $\KS\omega$~\cite{BaBarDKstar}. 
Also, both the BaBar and Belle collaborations have performed studies on \decay{\Bm}{\D\Kstarm} with \decay{\D}{\KS\pip\pim}~\cite{BaBarGGSZ,BelleGGSZ}.

\section{Analysis overview}
