% $Id: introduction.tex 87303 2016-02-08 13:44:29Z lafferty $
%\begin{savequote}[8cm]
%\textlatin{Neque porro quisquam est qui dolorem ipsum quia dolor sit amet, consectetur, adipisci velit...}
%
%There is no one who loves pain itself, who seeks after it and wants to have it, simply because it is pain...
%  \qauthor{--- Cicero's \textit{de Finibus Bonorum et Malorum}}
%\end{savequote}

\chapter{\label{ch:2-background}\CP violation and measurements of CKM angle \Pgamma} 

%\minitoc

\section{The Standard Model}

The Standard Model (SM) of particle physics describes all known fundamental particles and their interactions. Its predictions have been rigorously tested over several decades and have been found to be consistent with our observations to remarkable levels of precision~\cite{SMprecision}. Although the SM has proven to be extremely accurate it has several limitations, for example the model does not incorporate gravitational interactions, provide a dark matter candidate, or explain the huge asymmetry observed between matter and antimatter. These problems suggest that the SM is an incomplete theory, hence experimental particle physics is driven by the search for direct or indirect signs of New Physics.

The SM contains twelve spin-1/2 fermions in three generations, which are grouped into quarks and leptons, listed in \tab\ref{SMfermions}. All these elementary particles have corresponding antiparticles. The fermions interact via the electromagnetic, weak and strong forces, which are mediated by gauge bosons listed in \tab\ref{SMbosons}. The Higgs boson ($H$) is the only elementary scalar particle, and it couples to all particles with mass. The Higgs boson is not a Gauge boson, but a neutral scalar boson, which couples to all particles with mass. The Higgs mechanism explains why weak gauge bosons have mass, while the photon is massless. 

Quarks do not exist as free particles, but always exist in bound states with other quarks, referred to as hadrons. There are different types of hadrons: most commonly mesons and baryons, in addition to the recently discovered pentaquarks and tetraquarks~\cite{pentaquark,tetraquark_BESIII,tetraquark_lhcb}. Mesons are two-particle bound states composed of a quark and an antiquark, and baryons are three-particle bound states composed of three quarks. Examples of mesons include \B mesons, containing a \bquark or \bquarkbar quark (\eg~\Bm, \Bp, \Bz), and \D mesons, containing a \cquark or \cquarkbar quark (\eg~\Dm, \Dp, \Dz). Examples of baryons include protons, which are composed of \uquark\uquark\dquark quarks, and \Lz baryons, which are composed of \uquark\dquark\squark quarks. Tetraquarks consist of two quarks and two antiquarks, and pentaquarks consist of four quarks and one antiquark.

\begin{table}
\centering
\begin{tabular}{c|cc|cc}
& \multicolumn{2}{p{6cm}}{\hspace{2.2cm} Quarks} & \multicolumn{2}{p{6cm}}{\hspace{2.2cm} Leptons} \\
& Flavour & Charge & Flavour & Charge \\
\hline \hline
First generation & up (\uquark) & $+2/3$ & electron (e) & $-1$ \\
 & down (\dquark) & $-1/3$ & electron neutrino (\neue) & $0$ \\
\hline
Second generation & charm (\cquark) & $+2/3$ & muon (\muon) & $-1$ \\
 & strange (\squark) & $-1/3$ & muon neutrino (\neum) & $0$ \\
\hline
Third generation & top (\tquark) & $+2/3$ & tau (\tauon) & $-1$ \\
 & bottom (\bquark) & $-1/3$ & tau neutrino (\neut) & $0$ \\
\end{tabular}
\caption{Quarks and leptons in the Standard Model.}
\label{SMfermions}
\end{table}

\begin{table}
\centering
\begin{tabular}{c|cc}
Gauge boson & Force & Charge \\
\hline
photon (\g) & Electromagnetic & $0$ \\
\Z & Weak & $0$ \\
\Wpm & Weak & $\pm 1$ \\
gluon ($g$) & Strong & $0$ \\
\hline
$H$ (scalar) & - & $0$
\end{tabular}
\caption{Gauge and Higgs bosons in the Standard Model.}
\label{SMbosons}
\end{table}

\section{\CP violation}

One of the key problems with the SM is that it does not explain why our universe consists almost entirely of matter, with hardly any antimatter present. This asymmetry is commonly quantified by the baryon-antibaryon asymmetry, $\frac{n_{B} - n_{\bar{B}}}{n_{\g}} \approx \frac{n_{B}}{n_{\g}}$, where $n_{B}$, $n_{\bar{B}}$ and $n_{\g}$ are the baryon, anti-baryon and photon number densities respectively. Astrophysical observations have measured the baryon-antibaryon asymmetry to be of the order of $10^{-10}$~\cite{astrophysicalasy}. The requirements for this asymmetry to be generated from a symmetrical initial state, known as baryogenesis, were first proposed by Andrei Sakharov in 1967~\cite{sakharov}. The three Sakharov conditions are: baryon number violation, the violation of charge ($C$) symmetry and charge-parity (\CP) symmetry, and departure from thermal equilibrium. These components are required in all baryogenesis models. The definitions of the $C$ and $P$ symmetries are given by their operators:
\begin{itemize}
\item The charge conjugation operator, $\hat{C}$, converts particles into their antiparticles,
\item The parity operator, $\hat{P}$, reverses the spatial axes so all vectors change sign.
\end{itemize}
\CP symmetry is violated if the system changes under the combined $\hat{C}\hat{P}$ transformation.

The SM satisfies the \CP violation condition for baryogenesis, however the amount of asymmetry that can be generated from \CP violation in the SM is many orders of magnitude smaller than the asymmetry observed from astrophysical observations~\cite{SMasy}. New Physics models that can introduce new sources of \CP violation~\cite{BSMCP}, such as supersymmetric models, can be developed by theorists and tested at the LHC and other experiments. However, it is also necessary to make precise measurements of \CP violation in the SM to improve our understanding of matter-antimatter asymmetry and search for indirect signs of New Physics.

\CP violation is probed using many different processes. The different types of \CP violation in the SM can be split into three categories:
\begin{itemize}
\item \textbf{Direct \CP violation in decay}: this is when the rate of decay of a particle is not equal to the rate of the decay of the corresponding antiparticle \eg
\begin{equation*}
\Gamma\left(\decay{\Bm}{\D\Km}\right) \neq \Gamma\left(\decay{\Bp}{\D\Kp}\right) \text{ ,}
\end{equation*}
where \D represents a  superposition of \Dz and \Dzb mesons.
\item \textbf{Indirect \CP violation in mixing}: Neutral mesons can oscillate between their particle and antiparticle states, $\Bz \leftrightarrow \Bzb$, as shown in \fig\ref{mixing}. Indirect \CP violation occurs when \decay{\Bz}{\Bzb} and \decay{\Bzb}{\Bz} proceed at different rates, \eg
\begin{equation*}
\Gamma\left(\decay{\Bz}{\decay{\Bzb}{X}}\right) \neq \Gamma\left(\decay{\Bzb}{\decay{\Bz} \text{ ,}{\bar{X}}}\right)
\end{equation*}
where $X$ is some measured final state accessible to both \Bz and \Bzb.
\begin{figure}[h]
\centering
\includegraphics[width=0.5\linewidth]{figures/theory/mixing.pdf}
\caption{Feynman diagram of mixing between $\Bz_q$ and $\Bzb_q$, where $q$ represents a \dquark or \squark quark.}
\label{mixing}
\end{figure}
\item \textbf{\CP violation from the interference of mixing and decay}: This is best illustrated by an example. A \Bs meson (containing a \bquark quark and \squark antiquark) can decay to a \Dsm\Kp or \Dsp\Km final state. However, this process can also proceed via \decay{\Bs}{\Bsb} mixing, where the \Bsb decays to \Dsm\Kp or \Dsp\Km. These decay paths to the same final state interfere with each other, which results in the third form of \CP violation, illustrated by 
\begin{equation*}
\Gamma\left(\decay{\Bs}{\Dsm\Kp}\right) \neq \Gamma\left(\decay{\Bs}{\decay{\Bsb}{\Dsm\Kp}}\right)
\end{equation*}
\end{itemize} 

Each of these three types of \CP violation can be investigated. This thesis focuses on measurements of direct \CP violation within the Standard Model.

\section{The CKM matrix}

Quarks in the Standard Model can interact via the strong, weak or electromagnetic interactions. The weak interaction couples to a rotation of the flavour eigenstates.
Therefore, the eigenstates that take part in the weak interaction (weak eigenstates) are a mixture of the flavour eigenstates that hadronise to produce the observable meson states. The Cabibbo-Kobayashi-Maskawa (CKM) matrix, $V_{CKM}$, given in \eqn\ref{CKMmatrix}, describes the relationship between the weak eigenstates ($d'$, $s'$, $b'$) and flavour eigenstates (\dquark, \squark, \bquark) of the quarks. 
\begin{equation}
\left(
\begin{array}{c} d' \\ s' \\ b'  \end{array} \right) =
\begin{pmatrix} V_{ud} & V_{us} & V_{ub} \\ V_{cd} & V_{cs} & V_{cb} \\ V_{td} & V_{ts} & V_{tb} \end{pmatrix} \left( 
\begin{array}{c} d \\ s \\ b \end{array} \right) =
V_{CKM} \left( \begin{array}{c} d \\ s \\ b \end{array} \right)
\label{CKMmatrix}
\end{equation}
A given element of this matrix, $V_{ij}$, defines the coupling of a $j \to i$ quark transition. Similarly $V_{ij}^*$ defines the coupling of a $\bar{j} \to \bar{i}$ antiquark transition. By definition the CKM matrix is unitary, i.e. $V_{CKM}V_{CKM}^* = \mathds{1}$, assuming there are only three generations of quarks. The CKM matrix is a complex $3 \times 3$ matrix, which yields 18 parameters. The unitarity requirement, corresponding to nine complex equations, reduces the number of free parameters, and five strong phases can be absorbed into the quark fields as they are not physically observable. This leaves four independent free parameters to describe the CKM matrix: three amplitudes and one phase. This free phase parameter is the source of \CP violation in the SM. 

A standard representation of the CKM matrix uses 3 angles, $\theta_{12}$, $\theta_{23}$ and $\theta_{13}$, and one \CP violating phase, $\delta$, as shown in \eqn\ref{standard}. These angles are defined and labelled in a way which relates to the mixing of two specific generations; couplings between the quark generation $i$ and $j$ vanish if $\theta_{ij} = 0$, and $s_{ij}$ and $c_{ij}$ represent $\sin\theta_{ij}$ and $\cos\theta_{ij}$ respectively.
\begin{equation}
V_{CKM} = \begin{pmatrix} 1 & 0 & 0 \\ 
0 & c_{23} & s_{23} \\ 
0 & -s_{23} & c_{23} \end{pmatrix}
\begin{pmatrix} c_{13} & 0 & s_{13}e^{-i\delta_{13}} \\ 
0 & 1 & 0 \\ 
-s_{13}e^{i\delta_{13}} & 0 & c_{13} \end{pmatrix}
\begin{pmatrix} c_{12} & s_{12} & 0 \\ 
-s_{12} & c_{12} & 0 \\ 
0 & 0 & 1 \end{pmatrix}
\label{standard}
\end{equation}

If the CKM matrix was equivalent to the identity matrix there would be no cross-generation weak interaction of the quarks \eg\ a \uquark quark transition mediated by a W boson could only result in a \dquark, not a \squark or \bquark quark. From empirical determination, the magnitude of the elements in the CKM matrix are~\cite{PDG2016}:
\begin{equation}
| V_{CKM} | = \begin{pmatrix} 0.97434^{+0.00011}_{0.00012} & 0.22506 \pm 0.00050 & 0.00357 \pm 0.00015 \\ 0.22492 \pm 0.00050 & 0.97351 \pm 0.00013 & 0.0411 \pm 0.0013 \\ 0.00875^{+0.00032}_{-0.00033} & 0.0403 \pm 0.0013 & 0.99915 \pm 0.00005 \end{pmatrix} \text{ .}
\end{equation}
It can be seen that quark transitions within the same generation are highly favoured; these are known as Cabibbo-favoured decays. Quark transitions across one generation are suppressed, and the suppression is even stronger across two generations. These are known as Cabibbo-suppressed transitions. 

The structure in the CKM matrix can be illustrated by the Wolfenstein parameterisation, given in \eqn\ref{wolf}, which uses parameters $A$, $\lambda$, $\rho$ and $\eta$. 
\begin{equation}
V_{CKM} = \begin{pmatrix} 1 - \lambda^2/2 & \lambda & A\lambda^3(\rho - i\eta) \\ 
-\lambda & 1 - \lambda^2/2 & A\lambda^2 \\ 
A\lambda^3(1 - \rho - i\eta) & -A\lambda^2 & 1 \end{pmatrix}
\label{wolf} \text{ .}
\end{equation} 
This is an approximation of the standard parameterisation given in \eqn\ref{standard}, expanded in powers of the relatively small parameter $\lambda = \sin\theta_{12} = 0.22$. The other parameters are defined by $A\lambda^2 = s_{23}$ and $A\lambda^3(\rho - i\eta) = s_{13}e^{-i\delta}$. The \CP violation can be determined by measuring $\rho - i\eta$.

\section{The CKM unitarity triangle}

Verifying the unitarity of the CKM matrix is of the utmost importance since non-unitarity is a clear sign of physics Beyond the Standard Model~\cite{CKMtriangle}. The unitarity of the CKM matrix leads to nine unitarity conditions, for example $\Vud\Vubs + \Vcd\Vcbs + \Vtd\Vtbs = 0$, which is the most readily applicable to \B physics. These relations can be represented as a triangle in the complex plane, as shown in \fig\ref{triangle}. The triangle representing the condition $\Vud\Vubs + \Vcd\Vcbs + \Vtd\Vtbs = 0$ is the most interesting as all the quantities are experimentally measurable and are of reasonable relative size. The angles are defined as $\alpha$, $\beta$ and \Pgamma, and the area of the triangle is proportional to the amount of \CP violation in the quark sector of the SM~\cite{CKMtriangle}. 
\begin{figure}
\centering
\includegraphics[trim = 50mm 50mm 50mm 50mm,clip,width=0.6\linewidth]{figures/theory/triangle.pdf}
\caption{Graphical representation, in the complex plane, of the unitarity condition $\Vud\Vubs + \Vcd\Vcbs + \Vtd\Vtbs = 0$, forming the CKM triangle with angles $\alpha$, $\beta$ and \Pgamma.}
\label{triangle}
\end{figure}

The values of $\alpha$, $\beta$ and \Pgamma can be accessed by looking at various \B meson decays. The current best measurements for these values are $\alpha = \left(87.6^{+3.5}_{-3.3}\right)^{\circ}$, $\beta = 21.85^{+0.68}_{-0.67}$ and $\Pgamma = \left(72.2^{+6.8}_{-7.3}\right)^{\circ}$~\cite{PDG2016,LHCb-PAPER-2016-032}. Overconstraining this unitarity triangle allows verification as to whether or not the triangle closes, i.e. whether it truly is a triangle. Obtaining inconsistent results from overconstraining would suggest that the CKM matrix is not unitary, which would be incompatible with the SM. 

The CKM angle $\Pgamma \equiv \arg\left(-\frac{\Vud{\Vub}^*}{\Vcd{\Vcb}^*}\right)$ is the angle with the largest uncertainty. This angle is measured using the \lhcb detector from the rates of charged and neutral \B decays to a \D meson (reconstructed in one of a variety of final states) and kaons or pions. These are known as direct measurements. In order to obtain the most precise direct measurement of \Pgamma, the individual measurements from each of these \Pgamma-sensitive decays are combined to produce a single value with a lower uncertainty. The latest published \lhcb combination is $\Pgamma = \left(72.2^{+6.8}_{-7.3}\right)^{\circ}$~\cite{LHCb-PAPER-2016-032}. This is the most precise determination of \Pgamma from direct measurements from a single experiment; other experiments are consistent, but less precise~\cite{Babar_gamma,Belle_gamma}. These measurements involve only tree level process, making them theoretically clean. New particles that are not predicted in the SM can only contribute at higher order, and are therefore highly suppressed. As \Pgamma can be determined at tree-level, direct measurements of this angle are dominated by SM contributions.

A global fit to the CKM triangle from CKMfitter~\cite{CKMFitter}, shown in \fig\ref{globalfit}, uses the current best measurements of various quantities, such as $\beta$, $\Delta m_d$ and $\Delta m_s$, as inputs, where $\Delta m_d$ and $\Delta m_s$ are the mass differences between the mass eigenstates of \Bz-\Bzb and \Bs-\Bsb respectively. When performing this fit any information on \Pgamma from direct measurements can be ignored. Assuming the SM, i.e. unitarity of CKM matrix, \Pgamma can be extracted from the global fit. This is known as an indirect measurement. The measurements for the inputs used include loop processes. The presence of loops in the corresponding Feynman diagrams allows additional Beyond the Standard Model diagrams, where new particles appear in the loop contributing to the amplitude at the same order. Therefore, these loop processes, and by extension the extracted indirect measurement of \Pgamma, are sensitive to New Physics. This method obtains a \Pgamma measurement of $(65.3^{+1.0}_{-2.5})^{\circ}$, where this determination of \Pgamma excludes all direct measurements and assumes the SM. The leading theoretical uncertainties on the indirect measurement are from lattice QCD, therefore these uncertainties are expected to decrease as lattice QCD calculations become more accurate. 
\begin{figure}[!ht]
\centering
\includegraphics[trim = 0mm 0mm 0mm 180mm,clip,width=0.9\linewidth]{figures/theory/rhoeta_small_global.pdf}
\caption{Diagram showing the current state of measurements of the unitarity triangle~\cite{CKMFitter}. The black line shows the best fit obtained by CKMfitter. The axes $\bar{\rho}$ and $\bar{\eta}$ are the normalised versions of the $\rho$ and $\eta$ parameters in the Wolfenstein parameterisation of the CKM matrix, shown in \eqn\ref{wolf}.}
\label{globalfit}
\end{figure}

The direct and indirect measurements of \Pgamma are currently consistent with each other, however, the result of \Pgamma from direct measurements has a relatively large uncertainty and higher central value. Therefore, improving the precision of the direct measurement of \Pgamma is necessary to verify whether or not the direct and indirect measurements are consistent, thereby testing the consistency of the SM. Improvements in the precision can be achieved through a combination of methods and measurements of various \B decays that are sensitive to \Pgamma; those relevant to this thesis are outlined below.

\section{Tree-level determination of \Pgamma using \decay{\Bpm}{\D K^{(*)\pm}} decays}
\label{sec:theory:gamma}

Direct measurements of \Pgamma can be made by exploiting the interference between \decay{\bquark}{\cquark\uquarkbar\squark} and \decay{\bquark}{\uquark\cquarkbar\squark} transitions. These transitions are present at tree-level in $\Bpm \to \D\kaon^{*\pm}$ decays, represented by the Feynman diagrams shown in \fig\ref{fig:B2DKstarmdiagram}, showing the \decay{\Bm}{\Dz\Kstarm} decay (left) and the \decay{\Bm}{\Dzb\Kstarm} decay (right). Here, as previously mentioned, \D represents the superposition of \Dz and \Dzb mesons. The branching
fraction is of similar order to \decay{\Bm}{\Dzb\Km} which has been extensively analysed~\cite{LHCb-PAPER-2016-003,LHCb-PAPER-2014-041,LHCb-PAPER-2015-014}.
\begin{figure}[!h]
\centering
\resizebox{0.48\linewidth}{!}{
	\begin{tikzpicture}[scale=0.98]
        % MAP OUT VERTICES (I have 8 of them)
        \coordinate (a) at (0,2); %b quark start
        \coordinate (b) at (0,1); %ubar quark start
        \coordinate (c) at (5,1); %ubar quark end
        \coordinate (d) at (5,2); %c quark end
        \coordinate (e) at (2,2); %start the W+
        \coordinate (f) at (5,2.67); %s quark
        \coordinate (g) at (5,3.33); %ubar quark
        \coordinate (h) at (4.0,3); %W end
        % DRAW LINES
        \draw[antiparticle] (b)  -- (c); %ubar quark
        \draw[particle] (a)  -- (e); %bbar quark
        \draw[particle] (e)  -- (d); %ubar quark 
        \draw[photon] (e) -- (h); %W+
        \draw[antiparticle] (h)  to (g); %W to ubar quark
        \draw[particle] (h)  to (f); %W to s quark 
        %DRAW LABELS
        \node at ($(a)$) [label={[label distance=-4mm] b},left] {};
        \node at ($(b)$) [label={[label distance=-4mm] $\bar{\rm u}$},left] {};
        \node at ($(c)$) [label={[label distance=-4mm] $\bar{\rm u}$},right]{};
        \node at ($(d)$) [label={[label distance=-4mm] c},right] {};
        \node at ($(f)$) [label={[label distance=-4mm] s},right] {};
        \node at ($(g)$) [label={[label distance=-4mm] $\bar{\rm u}$},right]{};
        \node at ($(e)$) [label={[xshift=25pt,yshift=10pt] $W^-$},above] {};

        %ADD BRACES
        \draw
        [black,decorate,decoration={brace,amplitude=5pt},xshift=-20pt,yshift=0pt]
          (0,0.8)  -- (0,2.2) node [black,midway,left=0pt,xshift=-5pt]{$\Bm$};
        \draw
        [black,decorate,decoration={brace,amplitude=5pt},xshift=20pt,yshift=0pt]
          (5,3.53) -- (5,2.47) node [black,midway,right=0pt,xshift=5pt]{$\Kstarm$};
         \draw 	[black,decorate,decoration={brace,amplitude=5pt},xshift=20pt,yshift=0pt]
          (5,2.2)  -- (5,0.8) node [black,midway,right=0pt,xshift=5pt] {$\Dz$};
            \end{tikzpicture}
        }
\resizebox{0.47\linewidth}{!}{
\begin{tikzpicture}[scale=0.98]
        % MAP OUT VERTICES (I have 8 of them)
        \coordinate (a) at (0,3); %b quark start
        \coordinate (b) at (0,1); %ubar quark start
        \coordinate (c) at (5,1); %ubar quark end
        \coordinate (d) at (5,3); %u quark end
        \coordinate (e) at (2,3); %start the W+
        \coordinate (f) at (5,1.67); %s quark
        \coordinate (g) at (5,2.33); %cbar quark
        \coordinate (h) at (4.0,2); %W end
        % DRAW LINES
        \draw[antiparticle] (b)  -- (c); %ubar quark
        \draw[particle] (a)  -- (e); %b quark
        \draw[particle] (e)  -- (d); %u quark 
        \draw[photon] (e) -- (h); %W+
        \draw[antiparticle] (h)  to (g); %W to cbar quark
        \draw[particle] (h)  to (f); %W to s quark 
        %DRAW LABELS
        \node at ($(a)$) [label={[label distance=-4mm] b},left] {};
        \node at ($(b)$) [label={[label distance=-4mm] $\bar{\rm u}$},left] {};
        \node at ($(c)$) [label={[label distance=-4mm] $\bar{\rm u}$},right]{};
        \node at ($(d)$) [label={[label distance=-4mm] u},right] {};
        \node at ($(f)$) [label={[label distance=-4mm] s},right] {};
        \node at ($(g)$) [label={[label distance=-4mm] $\bar{\rm c}$},right]{};
        \node at ($(e)$) [label={[xshift=25pt,yshift=-40pt] $W^-$},above] {};

        %ADD BRACES
        \draw 	[black,decorate,decoration={brace,amplitude=5pt},xshift=-20pt,yshift=0pt]
        (0,0.8)  -- (0,3.2) node [black,midway,left=0pt,xshift=-5pt] {$\Bm$};
        \draw [black,decorate,decoration={brace,amplitude=5pt},xshift=20pt,yshift=0pt]
        (5,1.85)  -- (5,0.8) node [black,midway,right=0pt,xshift=5pt] {$\Kstarm$};
         \draw [black,decorate,decoration={brace,amplitude=5pt},xshift=20pt,yshift=0pt]
        (5,3.2)  -- (5,2.15) node [black,midway,right=0pt,xshift=5pt] {$\bar{\rm \Dz}$};
        \end{tikzpicture}
    }
    \caption{Leading order Feynman diagrams for \decay{\Bm}{\Dz\Kstarm} (left) and \decay{\Bm}{\Dzb\Kstarm} (right).}
    \label{fig:B2DKstarmdiagram}
\end{figure}

The ratio of the amplitudes between the \decay{\Bm}{\Dzb\Kstarm} decay and the \decay{\Bm}{\Dz\Kstarm} decay, and their charge conjugates are given by,
\begin{equation}
\frac{\mathcal{A}\left(\decay{\Bm}{\Dzb\Kstarm}\right)}{\mathcal{A}\left(\decay{\Bm}{\Dz\Kstarm}\right)} = \rb^{DK^*} e^{i(\deltab^{DK^*} - \gamma)} \text{ , }
\frac{\mathcal{A}\left(\decay{\Bp}{\Dz\Kstarp}\right)}{\mathcal{A}\left(\decay{\Bp}{\Dzb\Kstarp}\right)} = \rb^{DK^*} e^{i(\deltab^{DK^*} + \gamma)} \text{ .}
\label{ratiodiagrams}
\end{equation}
There are three parameters in \eqn\ref{ratiodiagrams}: $\rb^{DK^*}$, $\deltab^{DK^*}$ and \Pgamma. Here $\rb^{DK^*}$ is the magnitude of the ratio of the amplitudes and $\deltab^{DK^*}$ is difference in strong phase between the \decay{\Bm}{\Dz\Kstarm} and \decay{\Bm}{\Dzb\Kstarm} decays. When the \D meson is reconstructed in a final state accessible to both \Dz and \Dzb meson states, $f(D)$, interference occurs, as shown in \fig\ref{paths}. This interference gives sensitivity to the weak phase \Pgamma.

\begin{figure}
\centering
%\includegraphics[trim = 20mm 120mm 100mm 20mm,clip,width=0.7\linewidth]{figures/theory/test.pdf}
\includegraphics[trim = 0mm 120mm 0mm 30mm,clip,width=\linewidth]{figures/theory/pathDiagrams.pdf}
\put(-425,75) {\small $\rb^{DK^*} e^{i(\deltab^{DK^*} - \gamma)}$}
\put(-215,75) {\small $\rb^{DK^*} e^{i(\deltab^{DK^*} + \gamma)}$}
\put(-285,70) {\small $\bar{A_{D}}$}
\put(-285,12) {\small $A_D$}
\put(-75,70) {\small $A_D$}
\put(-75,12) {\small $\bar{A_{D}}$}
\caption{Diagram of the interfering amplitudes of \decay{\Bm}{\D\Kstarm} (left) and \decay{\Bp}{\D\Kstarp} (right), where $A_D$ is the amplitude of \decay{\Dz}{f(D)} and $\bar{A_{D}}$ is the amplitude of \decay{\Dzb}{f(D)}.}
\label{paths}
\end{figure}

In the following description we assume \CP violation in the charm sector is negligible, and effects of \D mixing are neglected~\cite{charmcpv,charmmixing}. The partial widths for the \Bm and \Bp decays are given by \eqns\ref{pwminus} and \ref{pwplus}, where $A_D$ is the amplitude of \decay{\Dz}{f(D)} and $\bar{A_{D}}$ is the amplitude of \decay{\Dzb}{f(D)}.
\begin{align}
\Gamma\left(\decay{\Bm}{[f(D)]\Kstarm}\right) &\propto |A_D|^2 + (\rb^{DK^*})^2 |\bar{A_{D}}|^2 + 2\rb^{DK^*} Re\left[ A_D \bar{A_{D}} e^{-i(\deltab^{DK^*} - \gamma)} \right] \label{pwminus} \\
\Gamma\left(\decay{\Bp}{[f(D)]\Kstarp}\right) &\propto |A_D|^2 + (\rb^{DK^*})^2 |\bar{A_{D}}|^2 + 2\rb^{DK^*} Re\left[ A_D \bar{A_{D}} e^{-i(\deltab^{DK^*} + \gamma)} \right] \label{pwplus}
\end{align}

The values of the complex amplitudes, $A_{D}$ and $\bar{A_{D}}$, depend on the \D meson final state chosen. The sensitivity to \Pgamma can therefore be maximised by a judicious choice of \Dz decay mode, through the dependencies of $A_{D}$ and $\bar{A_{D}}$.

\subsection{The GLW and quasi-GLW methods}
\label{sec:theory:glw}

The theorists Gronau, London and Wyler proposed the study of \decay{\Bm}{\D\Km} modes with the \D decays into a \CP eigenstate, referred to as the GLW method~\cite{GL,GW}, such as the eigenstates \decay{\Dz}{\Kp\Km} and \decay{\Dz}{\pip\pim}. A \CP eigenstate is a state which is preserved under a \CP transformation. As these final states are \CP-even, $A_{D} = \bar{A_{D}}$, and hence the expressions from \eqns\ref{pwminus} and \ref{pwplus} can be simplified to
\begin{align}
\Gamma\left(\decay{\Bm}{[f_{GLW}]\Kstarm}\right) &\propto 1 + (\rb^{DK^*})^2 + 2\rb^{DK^*}\cos(\deltab^{DK^*} - \gamma) \label{widthBm} \\
\Gamma\left(\decay{\Bp}{[f_{GLW}]\Kstarp}\right) &\propto 1 + (\rb^{DK^*})^2 + 2\rb^{DK^*}\cos(\deltab^{DK^*} + \gamma) \text { ,}\label{widthBp}
\end{align}
assuming that \CP violation in \D decays is negligible and $\mathcal{A}\left(\decay{\Bm}{\Dz\Kstarm}\right) = \mathcal{A}\left(\decay{\Bp}{\Dzb\Kstarp}\right)$. 

The four-body \D decay mode \decay{\D}{\pip\pim\pip\pim} is a self-conjugate decay mode, containing a mixture of \CP-even and \CP-odd states, which can be used to measure \Pgamma via the GLW method provided the fractional \CP-even content is known~\cite{NAYAK20151}. As this mode is not a pure \CP eigenstate, it is referred to as a quasi-GLW (qGLW) mode, and its sensitivity to \Pgamma is reduced. The \CP-even fraction, $F_{4\pi}$, measured to be $0.737 \pm 0.028$~\cite{charm4pi}, accounts for the dilution effect. The partial widths for this self-conjugate qGLW mode~\cite{NAYAK20151,charm4pi}, corresponding to \eqns\ref{widthBm} and \ref{widthBp}, are
\begin{align}
\Gamma\left(\decay{\Bm}{[f_{qGLW}]\Kstarm}\right) \propto 1 + (\rb^{DK^*})^2 + 2\rb^{DK^*}\left(2F_{4\pi} - 1\right)\cos(\deltab^{DK^*} - \gamma) \label{widthBm4body} \\
\Gamma\left(\decay{\Bp}{[f_{GLW}]\Kstarp}\right) \propto 1 + (\rb^{DK^*})^2 + 2\rb^{DK^*}\left(2F_{4\pi} - 1\right)\cos(\deltab^{DK^*} + \gamma) \text { .} \label{widthBp4body}
\end{align}
In \eqns\ref{widthBm4body} and \ref{widthBp4body}, it can be seen that the expression $2F_{4\pi} - 1$ modulates the \Pgamma-sensitive interference term.


\subsection{The ADS method}
\label{sec:theory:ads}

The theorists Atwood, Dunietz and Soni proposed looking at \D decay modes where $f(D)$ is a non-\CP eigenstate, \eg~\decay{\D}{\Km\pip}, referred to as the ADS method~\cite{ADS,ADS-2001}. A key feature of this method is that although the \Dz and \Dzb decay to the same final state, they proceed by very different amplitudes. The Feynman diagrams for the doubly Cabibbo-favoured \decay{\Dz}{\Km\pip} decay and the doubly Cabibbo-suppressed \decay{\Dz}{\Kp\pim} decay are shown in \fig\ref{fig:D2KPidiagram}.
\begin{figure}[!h]
\centering
\resizebox{0.48\linewidth}{!}{
	\begin{tikzpicture}[scale=0.98]
        % MAP OUT VERTICES (I have 8 of them)
        \coordinate (a) at (0,2); %c quark start
        \coordinate (b) at (0,1); %ubar quark start
        \coordinate (c) at (5,1); %ubar quark end
        \coordinate (d) at (5,2); %s quark end
        \coordinate (e) at (2,2); %start the W+
        \coordinate (f) at (5,2.67); %dbar quark
        \coordinate (g) at (5,3.33); %u quark
        \coordinate (h) at (4.0,3); %W end
        % DRAW LINES
        \draw[antiparticle] (b)  -- (c); %ubar quark
        \draw[particle] (a)  -- (e); %bbar quark
        \draw[particle] (e)  -- (d); %ubar quark 
        \draw[photon] (e) -- (h); %W+
        \draw[antiparticle] (h)  to (g); %W to ubar quark
        \draw[particle] (h)  to (f); %W to s quark 
        %DRAW LABELS
        \node at ($(a)$) [label={[label distance=-4mm] c},left] {};
        \node at ($(b)$) [label={[label distance=-4mm] $\bar{\rm u}$},left] {};
        \node at ($(c)$) [label={[label distance=-4mm] $\bar{\rm u}$},right]{};
        \node at ($(d)$) [label={[label distance=-4mm] s},right] {};
        \node at ($(f)$) [label={[label distance=-4mm] $\bar{\rm d}$},right]{};
        \node at ($(g)$) [label={[label distance=-4mm] u},right]{};
        \node at ($(e)$) [label={[xshift=25pt,yshift=10pt] $W^+$},above] {};

        %ADD BRACES
        \draw
        [black,decorate,decoration={brace,amplitude=5pt},xshift=-20pt,yshift=0pt]
          (0,0.8)  -- (0,2.2) node [black,midway,left=0pt,xshift=-5pt]{\Dz};
        \draw
        [black,decorate,decoration={brace,amplitude=5pt},xshift=20pt,yshift=0pt]
          (5,3.53) -- (5,2.47) node [black,midway,right=0pt,xshift=5pt]{\pip};
         \draw 	[black,decorate,decoration={brace,amplitude=5pt},xshift=20pt,yshift=0pt]
          (5,2.2)  -- (5,0.8) node [black,midway,right=0pt,xshift=5pt] {\Km};
            \end{tikzpicture}
        }
\resizebox{0.47\linewidth}{!}{
\begin{tikzpicture}[scale=0.98]
        % MAP OUT VERTICES (I have 8 of them)
        \coordinate (a) at (0,2); %c quark start
        \coordinate (b) at (0,1); %ubar quark start
        \coordinate (c) at (5,1); %ubar quark end
        \coordinate (d) at (5,2); %d quark end
        \coordinate (e) at (2,2); %start the W+
        \coordinate (f) at (5,2.67); %u quark
        \coordinate (g) at (5,3.33); %sbar quark
        \coordinate (h) at (4.0,3); %W end
        % DRAW LINES
        \draw[antiparticle] (b)  -- (c); %ubar quark
        \draw[particle] (a)  -- (e); %bbar quark
        \draw[particle] (e)  -- (d); %ubar quark 
        \draw[photon] (e) -- (h); %W+
        \draw[antiparticle] (h)  to (g); %W to ubar quark
        \draw[particle] (h)  to (f); %W to s quark 
        %DRAW LABELS
        \node at ($(a)$) [label={[label distance=-4mm] c},left] {};
        \node at ($(b)$) [label={[label distance=-4mm] $\bar{\rm u}$},left] {};
        \node at ($(c)$) [label={[label distance=-4mm] $\bar{\rm u}$},right]{};
        \node at ($(d)$) [label={[label distance=-4mm] d},right]{};
        \node at ($(f)$) [label={[label distance=-4mm] u},right]{};
        \node at ($(g)$) [label={[label distance=-4mm] $\bar{\rm s}$},right]{};
        \node at ($(e)$) [label={[xshift=25pt,yshift=10pt] $W^+$},above] {};

        %ADD BRACES
        \draw
        [black,decorate,decoration={brace,amplitude=5pt},xshift=-20pt,yshift=0pt]
          (0,0.8)  -- (0,2.2) node [black,midway,left=0pt,xshift=-5pt]{\Dz};
        \draw
        [black,decorate,decoration={brace,amplitude=5pt},xshift=20pt,yshift=0pt]
          (5,3.53) -- (5,2.47) node [black,midway,right=0pt,xshift=5pt]{\Kp};
         \draw 	[black,decorate,decoration={brace,amplitude=5pt},xshift=20pt,yshift=0pt]
          (5,2.2)  -- (5,0.8) node [black,midway,right=0pt,xshift=5pt] {\pim};
            \end{tikzpicture}
    }
    \caption{Leading order Feynman diagrams for \decay{\Dz}{\Km\pip} (left) and \decay{\Dz}{\Kp\pim} (right).}
    \label{fig:D2KPidiagram}
\end{figure}

The ratio of the amplitudes of \decay{\Dzb}{\Km\pip} and \decay{\Dz}{\Km\pip} is given by\footnote{The \CP operator acting on the state $\ket{\Dz}$ can result in either $\CP\ket{\Dz} = \ket{\Dzb}$ or $\CP\ket{\Dz} = - \ket{\Dzb}$. This has consequences for the definition of the strong phase difference. In the ADS formalism, which is used here, the definition $\CP\ket{\Dz} = \ket{\Dzb}$ is used, resulting in \eqn\ref{ratioads}. The alternative definition, $\CP\ket{\Dz} = - \ket{\Dzb}$, used by HFLAV and in the CLEO-c analysis in Ref~\cite{charmkpi_deltab}, would result in $r_D^{K\pi}e^{-i(\pi + \delta_D^{K\pi})}$. The consequence of this is that the measured value of the strong phase difference taken from Ref.~\cite{charmkpi_deltab} must be offset by $180^{\circ}$ when used in the analysis presented in this thesis.}
\begin{equation}
\frac{\mathcal{A}\left(\decay{\Dzb}{\Km\pip}\right)}{\mathcal{A}\left(\decay{\Dz}{\Km\pip}\right)} = r_D^{K\pi}e^{-i\delta_D^{K\pi}} \text { ,}
\label{ratioads}
\end{equation}
where the parameter $r_D^{K\pi}$ is the magnitude of the ratio of amplitudes and $\delta_D^{K\pi}$ is the difference in strong phase between the suppressed and favoured decays. These parameters have been previously measured to be $r_D^{K\pi} = 0.0591 \pm 0.0003$ and $\delta_D^{K\pi} = \left(191.8 \pm 12.1\right)^{\circ}$~\cite{HFAG}.

\Eqn\ref{ratioads} can be expressed as $\bar{A_{D}}/A_{D} = r_D^{K\pi}e^{-i\delta_D^{K\pi}}$ for the case where $f(D)$ is \Km\pip, and similarly, $A_{D}/\bar{A_{D}} = r_D^{K\pi}e^{-i\delta_D^{K\pi}}$ for the case where $f(D)$ is \Kp\pim. Using these expressions, the partial widths from \eqns\ref{pwminus} and \ref{pwplus} for the various \decay{\Bpm}{\D\Kstarpm} ADS decay modes are given by
\begin{align}
\Gamma\left(\decay{\Bm}{[\Kp\pim]\Kstarm}\right) &\propto (\rb^{DK^*})^2 + (r_D^{K\pi})^2 + 2\rb^{DK^*}r_D^{K\pi}\cos(\deltab^{DK^*} + \delta_D^{K\pi} - \gamma) \label{adsBdecaym} \\
\Gamma\left(\decay{\Bp}{[\Km\pip]\Kstarp}\right) &\propto (\rb^{DK^*})^2 + (r_D^{K\pi})^2 + 2\rb^{DK^*}r_D^{K\pi}\cos(\deltab^{DK^*} + \delta_D^{K\pi} + \gamma) \label{adsBdecayp} \\
\Gamma\left(\decay{\Bm}{[\Km\pip]\Kstarm}\right) &\propto 1 + (\rb^{DK^*})^2(r_D^{K\pi})^2 + 2\rb^{DK^*}r_D^{K\pi}\cos(\deltab^{DK^*} + \delta_D^{K\pi} + \gamma) \label{favBdecaym} \\
\Gamma\left(\decay{\Bp}{[\Kp\pim]\Kstarp}\right) &\propto 1 + (\rb^{DK^*})^2(r_D^{K\pi})^2 + 2\rb^{DK^*}r_D^{K\pi}\cos(\deltab^{DK^*} + \delta_D^{K\pi} - \gamma) \text {.} \label{favBdecayp} 
\end{align}
For the analysis considered in this thesis, the favoured decay contains a \Kstarm meson and pion from the \D meson of opposite charge, while in the suppressed decay the \Kstarm meson and the pion from the \D meson have the same charge. The ADS decay mode is a combination of a CKM-favoured \decay{\Bm}{\Dz\Kstarm} decay (\fig\ref{fig:B2DKstarmdiagram}, left), followed by a doubly Cabibbo-suppressed \decay{\Dz}{\Kp\pim} decay (\fig\ref{fig:D2KPidiagram}, right), and a CKM- and colour-suppressed \decay{\Bm}{\Dzb\Kstarm} decay (\fig\ref{fig:B2DKstarmdiagram}, right), followed by a Cabibbo-favoured \decay{\Dzb}{\Kp\pim} decay (\fig\ref{fig:D2KPidiagram}, left). Both paths to the same final state have amplitudes of similar size, and interference effects are therefore magnified in comparison to the GLW decay modes, where the decay path via the CKM-favoured \decay{\Bm}{\Dz\Kstarm} decay dominates.

Two-body \decay{\D}{\Kmp\pipm} decays are characterised by two parameters, an amplitude ratio and a single strong phase, as illustrated in \eqn\ref{ratioads}. However for multibody \decay{\D}{\Kmp\pipm\pimp\pipm} decays, the strong phase varies over the phase space. Therefore, the amplitude at each point in phase space, $p$, must be considered for both the \decay{\Dz}{\Km\pip\pim\pip} favoured and \decay{\Dz}{\Kp\pim\pip\pim} suppressed decays, referred to as $A_{fav}(p)$ and $A_{sup}(p)$ respectively. The three hadronic parameters that describe the \decay{\D}{\Km\pip\pim\pip} decay~\cite{charmk3pi,charmk3pi_errata,LHCb-PAPER-2015-057} are defined by
\begin{align}
r_D^{K3\pi} &= \frac{\int \mathrm{d}p \left|A_{sup}(p)\right|^2}{\int \mathrm{d}p \left|A_{fav}(p)\right|^2}
\label{rddefinition}
\end{align}
and
\begin{align}
R_{K3\pi} e^{i\delta_D^{K3\pi}} &= \frac{\int \mathrm{d}p A_{fav}(p)A_{sup}(p)}{\sqrt{\int \mathrm{d}p \left|A_{fav}(p)\right|^2 \int \mathrm{d}p \left|A_{sup}(p)\right|^2}} \text { ,}
\label{Rddefinition}
\end{align}
where $r_D^{K3\pi}$ is the amplitude ratio between suppressed and favoured \decay{\D}{\Km\pip\pim\pip} decays (analogous to $r_D^{K\pi}$), $R_{K3\pi}$ is a coherence factor to account for the dilution of the interference effects due to averaging over phase space, and $\delta_D^{K3\pi}$ is the strong phase difference (analogous to $\delta_D^{K\pi}$). These parameters have been measured to be $r_D^{K3\pi} = 0.0549 \pm 0.0006$, $R_{K3\pi} = 0.43 \pm 0.17$ and $\delta_D^{K3\pi} = \left(128 \pm 28\right){\circ}$~\cite{charmk3pi,charmk3pi_errata,LHCb-PAPER-2015-057}. Based on the definitions in \eqns\ref{rddefinition} and \ref{Rddefinition}, the partial widths are given by
\begin{align}
\Gamma\left(\decay{\Bm}{[\Kp\pim]\Kstarm}\right) &\propto (\rb^{DK^*})^2 + (r_D^{K3\pi})^2 + 2\rb^{DK^*}R_{K3\pi}r_D^{K3\pi}\cos(\deltab^{DK^*} + \delta_D^{K3\pi} - \gamma) \label{adsBdecaym4body} \\
\Gamma\left(\decay{\Bp}{[\Km\pip]\Kstarp}\right) &\propto (\rb^{DK^*})^2 + (r_D^{K3\pi})^2 + 2\rb^{DK^*}R_{K3\pi}r_D^{K3\pi}\cos(\deltab^{DK^*} + \delta_D^{K3\pi} + \gamma) \label{adsBdecayp4body} \\
\Gamma\left(\decay{\Bm}{[\Km\pip]\Kstarm}\right) &\propto 1 + (\rb^{DK^*})^2(r_D^{K3\pi})^2 + 2\rb^{DK^*}R_{K3\pi}r_D^{K3\pi}\cos(\deltab^{DK^*} + \delta_D^{K3\pi} + \gamma) \label{favBdecaym4body} \\
\Gamma\left(\decay{\Bp}{[\Kp\pim]\Kstarp}\right) &\propto 1 + (\rb^{DK^*})^2(r_D^{K3\pi})^2 + 2\rb^{DK^*}R_{K3\pi}r_D^{K3\pi}\cos(\deltab^{DK^*} + \delta_D^{K3\pi} - \gamma) \text{ .}\label{favBdecayp4body} 
\end{align}
It can be seen from these equations that the four-body coherence factor, $R_{K3\pi}$, modulates the size of the interference term that carries the dependence on \Pgamma.

\subsection{Physical observables}

This thesis aims to measure \Pgamma as well as the hadronic parameters of \btodkst decays, with the \Kstarm meson decaying to \KS\pim, using the methods and equations discussed in Secs.~\ref{sec:theory:glw} and \ref{sec:theory:ads}. However, experimental considerations must be taken into account when developing the strategy to measure these parameters. For example, the use of the parameters $\rb^{DK^*}$ and $\deltab^{DK^*}$ discussed in \sect\ref{sec:theory:gamma} assumes that a pure sample of \btodkst decays is available. This section introduces the \btodkst coherence factor, $\kappa$, which deals with the contamination from other \decay{\Bm}{\D\KS\pim} processes in order to extract the parameters of interest. This section also describes the experimental quantities that are measured in this analysis in order to gain maximum sensitivity to \Pgamma. 

\subsubsection{Coherence factor}
\label{sec:theory:kappa}

As discussed earlier in this chapter, this thesis considers \decay{\Bm}{\D\Kstarm}, \decay{\Kstarm}{\KS\pim} decays. The \Kstarm meson could be reconstructed in either of the \KS\pim or \Km\piz final states, however, due to the higher reconstruction efficiency of the \KS meson compared to the \piz meson, the \KS\pim final state is pursued. As the \Kstarm meson has a large natural width (about 50~\mevcc~\cite{PDG2016}), it is necessary to consider the effect of other resonant and non-resonant \decay{\Bm}{\D\KS\pim} decays on the experimental measurements of the physics parameters, $\rb^{DK^*}$, $\rb^{DK^*}$ and \Pgamma. The amplitudes and phases of the different decays will vary at each point in \decay{\Bm}{\D\KS\pim} phase space, as given by
\begin{align*}
A(B^- \to D^0 X^-) &= A_c(p) e^{i\delta_c(p)} \\
A(B^- \to \bar{D^0} X^-) &= A_u(p) e^{i(\delta_u(p) - \gamma)} \text{ ,}
\end{align*}
where $p$ is the point in $\left(m^2(\KS\pim),m^2(\D\pim)\right)$ space, where $m^2(\KS\pim) = (p_{\KS} + p_{\pim})^2$, and $m^2(\D\pim) = (p_{\D} + p_{\pim})^2$. Here $p_{X}$ is the four-momentum of particle $X$, the expressions $A_u(p)$ and $A_c(p)$ are the moduli of the \decay{\bquark}{\uquark} and \decay{\bquark}{\cquark} amplitudes respectively, while $\delta_{c}(p)$ and $\delta_{u}(p)$ represent the strong phases of the relevant decay amplitudes. The symbol $X^-$ represents a resonant or non-resonant \KS\pim pair, which could be produced by the decay of the \Kstarm meson or by other contributions to the \decay{\Bm}{\D\KS\pim} final state. The amplitudes of the \Bp decays can be expressed as 
\begin{align*}
A(\decay{\Bp}{\Dzb X^+}) &= A_c(p) e^{i\delta_c(p)} \\
A(\decay{\Bp}{\Dz X^+}) &= A_u(p) e^{i(\delta_u(p) + \gamma)} \text{ .}
\end{align*}
Due to the large natural width of the \Kstarm meson, interference may occur in the region near the \Kstarm mass between the signal \Kstarm decay amplitude and amplitudes due to other \decay{\Bm}{\D\KS\pim} contributions, for example higher \KS\pim resonances and non-resonant decays. The interfering contributions dilute the sensitivity to \Pgamma, which is quantified by the \btodkst coherence factor, $\kappa$, where $0 \leq \kappa \leq 1$, and $\kappa = 1$ denotes a pure $K^{*-}$ contribution giving maximum sensitivity to \Pgamma. The parameters \rb, \deltab and $\kappa$ are then defined for \btodkst decays as
\begin{align}
r_B^2 &= \frac{\Gamma(B^- \to \bar{D^0}X^-)}{\Gamma(B^- \to D^0X^-)} = \frac{\int \left|A_u(p)\right|^2 \mathrm{d}p}{\int \left|A_c(p)\right|^2 \mathrm{d}p}
\label{rbdefinition}
\end{align}
\begin{align}
\kappa e^{i\delta_B} &= \frac{\int \mathrm{d}p A_c(p)A_u(p)e^{i\delta(p)}}{\sqrt{\int \mathrm{d}p \left|A_u(p)\right|^2 \int \mathrm{d}p \left|A_c(p)\right|^2}} \text{ ,}
\label{kappadefinition}
\end{align}
where $p$ represents a point in phase space, $0 \leq \delta(p) \leq 2\pi$, and the integration is performed over a defined \Kstarm region. In \eqns\ref{rbdefinition} and \ref{kappadefinition}, the parameters \rb, \deltab and $\kappa$ depend on the region of the \decay{\Bm}{\D\KS\pim} phase space that is integrated over. In order to maximise sensitivity to \Pgamma an integration region should be chosen that finds the optimal working point between maximising the coherence factor and maximising the size of the data sample available.

As before, it is assumed that \CP violation in the charm sector is negligible, and effects of \D mixing are neglected~\cite{charmcpv,charmmixing}. The partial widths for the \Bm and \Bp decays are then given by \eqns\ref{partialwidthminus} and \ref{partialwidthplus}, where $A_D$ is the amplitude of \decay{\Dz}{f(D)} and $\bar{A_{D}}$ is the amplitude of \decay{\Dzb}{f(D)}.
\begin{align}
\frac{d\Gamma\left(\decay{\Bm}{[f(D)]X^-}\right)}{dp} &\propto | A_c(p) e^{i\delta_c(p)}A_{D} + A_u(p) e^{i(\delta_u(p) - \gamma)}\bar{A_{D}} |^2 \label{partialwidthminus} \\
\frac{d\Gamma\left(\decay{\Bp}{[f(D)]X^-}\right)}{dp} &\propto | A_u(p) e^{i(\delta_u(p) + \gamma)}\bar{A_{D}} + A_c(p) e^{i\delta_c(p)}A_{D} |^2 \label{partialwidthplus}
\end{align}
Expanding and integrating over the defined \Kstar region gives
\begin{align}
\Gamma\left(\decay{\Bm}{[f(D)]X^-}\right) &\propto |A_D|^2 + \rb^2 |\bar{A_{D}}|^2 + 2\kappa\rb Re\left[ A_D \bar{A_{D}} e^{-i(\deltab - \gamma)} \right] \label{BandDdecaysminus} \\
\Gamma\left(\decay{\Bp}{[f(D)]X^-}\right) &\propto |A_D|^2 + \rb^2 |\bar{A_{D}}|^2 + 2\kappa\rb Re\left[ A_D \bar{A_{D}} e^{-i(\deltab + \gamma)} \right] \label{BandDdecaysplus}
\end{align}

A comparison between \eqns\ref{pwminus} and \ref{pwplus} and \eqns\ref{BandDdecaysminus} and \ref{BandDdecaysplus} shows that in order to account for the fact that the \decay{\Bm}{\D\Kstarm} sample contains small contributions from other resonant and non-resonant \decay{\Bm}{\D\KS\pim} decays, the substitutions $(\rb^{DK^*})^2 \to \rb^2$, $\rb^{DK^*} \to \kappa\rb$ and $\deltab^{DK^*} \to \deltab$, are evident. In particular for $\kappa = 1$, \eqns\ref{pwminus} and \ref{pwplus} are recovered. The coherence factor, $\kappa$, that accounts for contributions from other resonant and non-resonant \decay{\Bm}{\D\KS\pim} decays, modulates the size of the interference term that carries the dependence on \Pgamma.

\subsubsection{\CP observables}
\label{sec:theory:observables}

Using the GLW, qGLW and ADS methods, discussed in Secs.~\ref{sec:theory:glw} and \ref{sec:theory:ads}, observable quantities are constructed that can be used to extract \rb, \deltab and \Pgamma. The decay rates for the GLW modes, in \eqns\ref{widthBm} - \ref{widthBp4body}, the two-body ADS modes, in \eqns\ref{adsBdecaym} - \ref{favBdecayp}, and the four-body ADS modes, in \eqns\ref{adsBdecaym4body} - \ref{favBdecayp4body}, can be measured directly by counting the number of observed events. However, by constructing ratios of these decay rates many experimental uncertainties will cancel, thus improving the precision of the results. 

The observables used in this thesis are the asymmetries between the \Bm and \Bp decay rates, as well as ratios of decay rates in comparison to the favoured modes for the different \Dz final states. No \CP asymmetry is expected in the two- and four-body favoured \Dz decay modes. The twelve quantities, collectively referred to as \CP observables, that are measured in this analysis are:

\begin{itemize}
\item{The \CP asymmetry for the favoured decay mode
\begin{equation}
A_{\kaon\pi} = \frac{\Gamma\left(\decay{\Bm}{\D(\Km\pip)\Kstarm}\right) - \Gamma\left(\decay{\Bp}{\D(\Kp\pim)\Kstarp}\right)}{\Gamma\left(\decay{\Bm}{\D(\Km\pip)\Kstarm}\right) + \Gamma\left(\decay{\Bp}{\D(\Kp\pim)\Kstarp}\right)} \text{ .}
\label{eqn:Akpi}
\end{equation}}
\item{The \CP asymmetry for the \decay{\D}{\Kp\Km} decay mode
\begin{equation}
A_{\kaon\kaon} = \frac{\Gamma\left(\decay{\Bm}{\D(\Kp\Km)\Kstarm}\right) - \Gamma\left(\decay{\Bp}{\D(\Kp\Km)\Kstarp}\right)}{\Gamma\left(\decay{\Bm}{\D(\Kp\Km)\Kstarm}\right) + \Gamma\left(\decay{\Bp}{\D(\Kp\Km)\Kstarp}\right)} \text{ . }
\label{eqn:Akk}
\end{equation}
}
\item{The \CP asymmetry for the \decay{\D}{\pip\pim} decay mode
\begin{equation}
A_{\pi\pi} = \frac{\Gamma\left(\decay{\Bm}{\D(\pip\pim)\Kstarm}\right) - \Gamma\left(\decay{\Bp}{\D(\pip\pim)\Kstarp}\right)}{\Gamma\left(\decay{\Bm}{\D(\pip\pim)\Kstarm}\right) + \Gamma\left(\decay{\Bp}{\D(\pip\pim)\Kstarp}\right)} \text{ . }
\label{eqn:Apipi}
\end{equation}}
\item{The ratio of the rate for the \decay{\D}{\Kp\Km} decay mode to that of the favoured decay mode, scaled by the branching fractions
\begin{multline}
R_{\kaon\kaon} = \frac{\Gamma\left(\decay{\Bm}{\D(\Kp\Km)\Kstarm}\right) + \Gamma\left(\decay{\Bp}{\D(\Kp\Km)\Kstarp}\right)}{\Gamma\left(\decay{\Bm}{\D(\Km\pip)\Kstarm}\right) + \Gamma\left(\decay{\Bp}{\D(\Kp\pim)\Kstarp}\right)} \\ \times \frac{\BR(D^0 \to K^-\pi^+)}{\BR(D^0 \to K^+K^-)} \text{ . }
\label{eqn:Rkk}
\end{multline}
}
\item{The ratio of the rate for the \decay{\D}{\pip\pim} decay mode to that of the favoured decay mode, scaled by the branching fractions
\begin{multline}
R_{\pi\pi} = \frac{\Gamma\left(\decay{\Bm}{\D(\pip\pim)\Kstarm}\right) + \Gamma\left(\decay{\Bp}{\D(\pip\pim)\Kstarp}\right)}{\Gamma\left(\decay{\Bm}{\D(\Km\pip)\Kstarm}\right) + \Gamma\left(\decay{\Bp}{\D(\Kp\pim)\Kstarp}\right)} \\ \times \frac{\BR(D^0 \to K^-\pi^+)}{\BR(D^0 \to \pi^+\pi^-)} \text{ . }
\label{eqn:Rpipi}
\end{multline}}
\item{The ratio of the rate for the ADS decay mode to that of the favoured decay mode for \Bp decays
\begin{equation}
R^+_{K\pi} = \frac{\Gamma\left(\decay{\Bp}{\D(\Km\pip)\Kstarp}\right)}{\Gamma\left(\decay{\Bp}{\D(\Kp\pim)\Kstarp}\right)} \text{ . }
\label{eqn:Rplus}
\end{equation}
}
\item{The ratio of the rate for the ADS decay mode to that of the favoured decay mode for \Bm decays
\begin{equation}
R^-_{K\pi} = \frac{\Gamma\left(\decay{\Bm}{\D(\Kp\pim)\Kstarm}\right)}{\Gamma\left(\decay{\Bm}{\D(\Km\pip)\Kstarm}\right)} \text{ . }
\label{eqn:Rminus}
\end{equation}
}
\item{The \CP asymmetry for the favoured \decay{\Dz}{\Km\pip\pim\pip} decay mode
{\footnotesize
\begin{equation}
A_{\kaon\pi\pi\pi} = \frac{\Gamma\left(\decay{\Bm}{\D(\Km\pip\pim\pip)\Kstarm}\right) - \Gamma\left(\decay{\Bp}{\D(\Kp\pim\pip\pim)\Kstarp}\right)}{\Gamma\left(\decay{\Bm}{\D(\Km\pip\pim\pip)\Kstarm}\right) + \Gamma\left(\decay{\Bp}{\D(\Kp\pim\pip\pim)\Kstarp}\right)} \text{ . }
\label{eqn:Akpipipi}
\end{equation}}}
\item{The \CP asymmetry for the \decay{\D}{\pip\pim\pip\pim} decay mode
{\footnotesize
\begin{equation}
A_{\pi\pi\pi\pi} = \frac{\Gamma\left(\decay{\Bm}{\D(\pip\pim\pip\pim)\Kstarm}\right) - \Gamma\left(\decay{\Bp}{\D(\pip\pim\pip\pim)\Kstarp}\right)}{\Gamma\left(\decay{\Bm}{\D(\pip\pim\pip\pim)\Kstarm}\right) + \Gamma\left(\decay{\Bp}{\D(\pip\pim\pip\pim)\Kstarp}\right)} \text{ . }
\label{eqn:Apipipipi}
\end{equation}
}}
\item{The ratio of the rate for the \decay{\D}{\pip\pim\pip\pim} decay mode to that of the favoured decay mode, scaled by the branching fractions
{\footnotesize
\begin{multline}
R_{\pi\pi\pi\pi} = \frac{\Gamma\left(\decay{\Bm}{\D(\pip\pim\pip\pim)\Kstarm}\right) + \Gamma\left(\decay{\Bp}{\D(\pip\pim\pip\pim)\Kstarp}\right)}{\Gamma\left(\decay{\Bm}{\D(\Km\pip\pim\pip)\Kstarm}\right) + \Gamma\left(\decay{\Bp}{\D(\Kp\pim\pip\pim)\Kstarp}\right)} \\
\times \frac{\mathcal{B}(D^0 \to \Km\pip\pim\pip)}{\mathcal{B}(D^0 \to \pip\pim\pip\pim)} \text{ . }
\label{eqn:Rpipipipi}
\end{multline}}}
\item{The ratio of the rate for the four-body ADS decay mode to that of the four-body favoured decay mode for \Bp decays
\begin{equation}
R^+_{K\pi\pi\pi} = \frac{\Gamma\left(\decay{\Bp}{\D(\Km\pip\pim\pip)\Kstarp}\right)}{\Gamma\left(\decay{\Bp}{\D(\Kp\pim\pip\pim)\Kstarp}\right)} \text{ . }
\label{eqn:Rplus4body}
\end{equation}
}
\item{The ratio of the rate of the four-body ADS decay mode to that of the four-body favoured decay mode for \Bm decays
\begin{equation}
R^-_{K\pi\pi\pi} = \frac{\Gamma\left(\decay{\Bm}{\D(\Kp\pim\pip\pim)\Kstarm}\right)}{\Gamma\left(\decay{\Bm}{\D(\Km\pip\pim\pip)\Kstarm}\right)} \text{ . }
\label{eqn:Rminus4body}
\end{equation}
}
\end{itemize}

\noindent
The asymmetries $A_{\kaon\pi}$ and $A_{\kaon\pi\pi\pi}$ should be essentially zero due to the very small interference expected in the configuration of \B and \D decays. Due to negligible direct \CP violation in \D decays~\cite{charmcpv}, the observables $A_{\kaon\kaon}$ and $A_{\pi\pi}$ should be equal and are often labelled together as $A_{\CP+}$; similarly the observables $R_{\kaon\kaon}$ and $R_{\pi\pi}$ should be equal and are labelled $R_{\CP+}.$\footnote{The analogous observables to $R_{\CP+}$ and $A_{\CP+}$ for the ADS mode are $R_{ADS}$ and $A_{ADS}$. However, $R_{ADS}$ and $A_{ADS}$ are not used here for the ADS decay mode, instead the ratios are measured separately for the positive and negative charges. The reason for this choice is that the uncertainty in $A_{ADS}$ depends on the value of $R_{ADS}$, therefore these observables are statistically dependent, raising problems for the low yields expected in the ADS mode. Hence the statistically independent observables $R^+_{K\pi}$ and $R^-_{K\pi}$ are preferred.} The observables \Rkk, \Rpipi and \Rpipipipi are scaled by the relevant branching fraction ratio, as can be seen from \eqns\ref{eqn:Rkk}, \ref{eqn:Rpipi} and \ref{eqn:Rpipipipi} respectively. This is done in order to construct a \CP observable that is independent of final state, i.e. only depends on hadronic parameters of the \Bm decay. 

The \CP observables measured in this analysis can be related to the physics parameters to be determined, namely \Pgamma, $r_B$ and $\delta_B$. Given there is a negligible effect from both charm mixing~\cite{charmmixing} and \CP violation in \D decays~\cite{charmcpv}, the relationships between the \CP observables and physics parameters are summarised by the following equations:

\begin{align}
A_{\CP+} &= \frac{2 \kappa r_B\sin\delta_B\sin\gamma}{1 + r_B^2 + 2 \kappa r_B\cos\delta_B\cos\gamma} \text{ ,}
\label{exp_Acp} \\
R_{\CP+} &= 1 + r_B^2 + 2 \kappa r_B\cos\delta_B\cos\gamma \text{ ,}
\label{exp_Rcp} \\
R^{\pm}_{K\pi} &= \frac{r_B^2 + \left(r_D^{K\pi}\right)^2 + 2\kappa r_B r_D^{K\pi} \cos(\delta_B + \delta_D^{K\pi} \pm \gamma)}{1 + r_B^2\left(r_D^{K\pi}\right)^2 + 2\kappa r_B r_D^{K\pi} \cos(\delta_B - \delta_D^{K\pi} \pm \gamma)} \text{ ,}
\label{exp_Rpm} \\
A_{\pi\pi\pi\pi} &= \frac{2 \kappa\left(2F_{4\pi} - 1\right) r_B\sin\delta_B\sin\gamma}{1 + r_B^2 + 2 \kappa\left(2F_{4\pi} - 1\right) r_B\cos\delta_B\cos\gamma} \text{ ,}
\label{exp_A4pi} \\
R_{\pi\pi\pi\pi} &= 1 + r_B^2 + 2 \kappa\left(2F_{4\pi} - 1\right) r_B\cos\delta_B\cos\gamma \text{ ,}
\label{exp_R4pi} \\
R^{\pm}_{K\pi\pi\pi} &= \frac{r_B^2 + \left(r_D^{K3\pi}\right)^2 + 2\kappa r_B \kappa_{K3\pi} r_D^{K3\pi} \cos(\delta_B + \delta_D^{K3\pi} \pm \gamma)}{1 + \left(r_Br_D^{K3\pi}\right)^2 + 2\kappa r_B \kappa_{K3\pi} r_D^{K3\pi} \cos(\delta_B - \delta_D^{K3\pi} \pm \gamma)} \text{ .}
\label{exp_Rpm4body}
\end{align}

These six relationships contain 10 unknown parameters: the three parameters of interest, namely \rb, \deltab and \Pgamma, the coherence factor relating to the \Bm decay, $\kappa$, which has a value specific for this analysis, and six parameters, namely $r_D^{K\pi}$, $\delta_D^{K\pi}$, $r_D^{K3\pi}$, $\delta_D^{K3\pi}$, $R_{K3\pi}$ and $F_{4\pi}$ describing the various \Dz decays, which have all been measured with relatively small uncertainties~\cite{charm4pi,charmk3pi,charmk3pi_errata,LHCb-PAPER-2015-057}. Therefore, the strategy taken in this thesis to extract \rb, \deltab and \Pgamma is to constrain the six parameters describing the various \Dz decays directly from existing measurements, which allows a more precise determination of the parameters of interest.

The angles \deltab and \Pgamma relate to the \CP observables via the trigonometric functions $\sin$ and $\cos$, as shown in \eqns\ref{exp_Acp} - \ref{exp_Rpm4body}. These relationships result in a two-fold ambiguity in the interpretation of the angles between $\theta$ or $180^{\circ} - \theta$, where $\theta$ corresponds to \deltab or \Pgamma. Therefore, the results are expected to have multiple solutions.


\section{Previous \Pgamma measurements with \decay{\Bpm}{\D K^{(*)\pm}} decays}

%{\color{red}{I am struggling with this paragraph}}
%Out of the three CKM angles, $\alpha$, $\beta$ and \Pgamma, the measurement of \Pgamma has the largest uncertainty, despite it being the only angle whose measurements are dominated by tree-level processes. This is due to the experimental considerations of the decays sensitive to each of the CKM angles. For example, $\beta$ is measured primarily using \decay{\Bz}{\jpsi\KS}, where \decay{\jpsi}{\mup}{\mun}, which is very easy to reconstruct due to the two charged leptons in the final state, and so can be reconstructed with a very high efficiency. However, \Pgamma is measured primarily from \decay{\Bm}{\D\Km} decays, which is a purely hadronic decay and so more difficult to reconstruct due to the possibility of misidentification of the kaons and pions. This was especially true at the B factories, namely the BaBar and Belle experiments, as they did not have the particle identification power of \lhcb. The extraction of $\alpha$ is obtained by performing measurements of \decay{\Bz}{\pip\pim} and \decay{\Bz}{\rhop\rhom}. These experimental difficulties are the reason for \Pgamma having a larger uncertainty than $\alpha$ and $\beta$.

The \decay{\Bm}{\D\Km} channel is a very important mode for tree level \Pgamma measurements, as it is straightforward to reconstruct and has a fairly high branching fraction of $3.7 \times 10^{-4}$~\cite{PDG2016}. This \decay{\Bm}{\D\Km} channel is thoroughly exploited at \lhcb, having yielded many \Pgamma-sensitive measurements using an extensive range of \D meson final states:
\begin{itemize}
\item \decay{\D}{\Kp\pim}, \Kp\Km, \pip\pim~\cite{LHCb-PAPER-2017-021},
\item \decay{\D}{\Kp\pim\pip\pim}, \pip\pim\pip\pim~\cite{LHCb-PAPER-2016-003},
\item \decay{\D}{\Kp\pim\piz}, \Kp\Km\piz, \pip\pim\piz~\cite{LHCb-PAPER-2015-014},
\item \decay{\D}{\KS\Kp\Km}, \KS\pip\pim~\cite{LHCb-PAPER-2014-041},
\item \decay{\D}{\KS\Kp\pim}~\cite{LHCb-PAPER-2013-068}.
\end{itemize}
A combination of these \Pgamma-sensitive \CP violation measurements performed by \lhcb, in addition to other modes, has yielded a combined result of $\Pgamma = \left(72.2^{+6.8}_{-7.3}\right)^{\circ}$~\cite{LHCb-PAPER-2016-032}, which is the most precise direct determination of \Pgamma to date.

Other modes used to constrain \Pgamma at \lhcb include, \decay{\Bz}{\D\Kstarz} decays and the time-dependent \decay{\Bs}{\Dspm\Kmp} decays yielding individual measurements of $\Pgamma = (71 \pm 20)^{\circ}$ and $\Pgamma = (115^{+28}_{-43})^{\circ}$ (mod $180^{\circ}$) respectively~\cite{LHCb-PAPER-2016-006,LHCb-PAPER-2014-038}. Previous \B-factory experiments, namely the BaBar and Belle experiments, have also produced constraints on the CKM angle \Pgamma from charged and neutral \B decays. For example, decays of \decay{\Bm}{\D\Km}, $\D^*\Km$, \D\Kstarm and neutral \B decays to $\D^{(*)}K^{(*)}$ have been reported~\cite{BabarGLW_latest,BabarADS_latest,BaBar-Gamma-2013,BaBarGGSZ,BaBar_B0,BelleGLW_latest,BelleADS_latest,BelleGGSZ}. A combination of measurements from BaBar and Belle experiments separately obtained results of $\Pgamma = (69^{+17}_{-16})^{\circ}$ and $\Pgamma = (68^{+15}_{-14})^{\circ}$ respectively, where the first uncertainty is statistical, the second is systematic and the third is model uncertainty~\cite{Babar_gamma,Belle_gamma}.

The \decay{\Bm}{\D\Kstarm} channel is analogous to the frequently used \decay{\Bm}{\D\Km} decay in its physical properties, as well as having a comparable branching fraction of $5.3 \times 10^{-4}$~\cite{PDG2016}. However, prior to the work presented in this thesis, the \decay{\Bm}{\D\Kstarm} channel had not been investigated at \lhcb. The \Kstarm decays almost exclusively to \Kz\pim and \Km\piz final states, both of which involve a neutral particle. This makes \decay{\Bm}{\D\Kstarm} decays much more difficult to reconstruct at \lhcb than \decay{\Bm}{\D\Km} decays, resulting in the \decay{\Bm}{\D\Kstarm} channel having a lower sensitivity to \Pgamma. However, the \decay{\Bm}{\D\Kstarm} channel contains fewer problematic backgrounds, and the unknown hadronic parameters of the \decay{\Bm}{\D\Kstarm} decay, \rb and \deltab, can be accessed. Furthermore, by measuring \Pgamma in as many different channels as possible, the uncertainty on the combined measurement of \Pgamma will be reduced, and the consistency of the result between channels can be verified. 

The \decay{\Bm}{\D\Kstarm} channel has previously been investigated by the BaBar collaboration using a variety of \CP-even, \CP-odd and non-\CP two-body \D decay modes, namely \Km\Kp, \pim\pip, \KS\piz, $\KS\phi$, $\KS\omega$ and \Km\pip~\cite{BaBarDKstar}. Their study produced confidence levels for \Pgamma, excluding $[85,99]^{\circ}$ at the two sigma level. Also, both the BaBar and Belle collaborations have performed studies on \decay{\Bm}{\D\Kstarm} with \decay{\D}{\KS\pip\pim}, obtaining results of $\Pgamma = (76 \pm 22 \pm 5 \pm 5)^{\circ}$ and $\Pgamma = (53^{+15}_{-18} \pm 3 \pm 9)^{\circ}$ respectively~\cite{BaBarGGSZ,BelleGGSZ}. In all these measurements the \Kstarm meson is reconstructed in the \KS\pim mode, with \decay{\KS}{\pip\pim}.

In this analysis, the \decay{\Bm}{\D\Kstarm} channel is investigated, with \decay{\Kstarm}{\KS\pim} and \decay{\KS}{\pim\pip}, where the \Dz meson in reconstructed in its decay to \Km\pip, \Kp\Km, \pip\pim, \Kp\pim, \Km\pip\pim\pip, \pip\pim\pip\pim and \Kp\pim\pip\pim final states. The \Kstarm meson could be reconstructed from its decay to \Km\piz, however, the \piz meson decays almost exclusively to two photons, which are difficult to reconstruct, and therefore the \piz meson has a much poorer reconstruction efficiency and larger uncertainty than the \KS meson. Therefore the \Km\piz mode is not pursued in this thesis.

\section{Analysis overview}

The work presented in this thesis measures the \CP observables in \decay{\Bm}{\D\Kstarm} decays as well as considering the interpretation of these observables in terms of \rb, \deltab and \Pgamma. The analysis strategy is first to develop a procedure, based on our understanding of the data, to select \decay{\Bm}{\D(\Km\pip)\Kstarm} and \kpipipi signal decays while removing unwanted background events, discussed in \sect\ref{sec:selection}. The same selection with a few adjustments can subsequently be applied to the other two- and four-body \D decay modes. After applying this selection, a fit model is developed to describe the \B mass distribution of the remaining \decay{\Bm}{\D(\Km\pip)\Kstarm} and \kpipipi candidates, detailed in \sect\ref{sec:massfit}. Using the fit model developed based on these modes, a simultaneous fit is then performed to all seven \D modes, separated by \B charge, to extract the \CP observables, detailed in Chapter \ref{ch:5-cpfit}. Chapter \ref{ch:6-interpretation} discusses a determination of $\kappa$ using \eqn\ref{kappadefinition} as well as the interpretation of the \CP observables, using \eqns\ref{exp_Acp} - \ref{exp_Rpm4body}, in terms of the physics parameters \rb, \deltab and \Pgamma.