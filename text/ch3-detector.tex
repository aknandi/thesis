\begin{savequote}[8cm]
\textlatin{Neque porro quisquam est qui dolorem ipsum quia dolor sit amet, consectetur, adipisci velit...}

There is no one who loves pain itself, who seeks after it and wants to have it, simply because it is pain...
  \qauthor{--- Cicero's \textit{de Finibus Bonorum et Malorum}}
\end{savequote}

\chapter{\label{ch:3-detector}Detector} 

\minitoc

\section{The \lhcb detector and online selection}
\label{sec:detector}

The \lhcb detector~\cite{Alves:2008zz,LHCb-DP-2014-002} is a single-arm forward spectrometer covering the \mbox{pseudorapidity} range $2<\eta <5$, designed for the study of particles containing \bquark or \cquark quarks. The detector includes a high-precision tracking system consisting of a silicon-strip vertex detector (\velo) surrounding the $pp$ interaction region, a large-area silicon-strip detector located upstream of a dipole magnet with a bending power of about $4{\mathrm{\,Tm}}$, and three stations of silicon-strip detectors and straw drift tubes placed downstream of the magnet. The tracking system provides a measurement of momentum, \ptot, of charged particles with a relative uncertainty that varies from 0.5\% at low momentum to 1.0\% at 200\gevc. The minimum distance of a track to a primary vertex (PV), the impact parameter (IP), is measured with a resolution of $(15+29/\pt)\mu m$, where \pt is the component of the momentum transverse to the beam, in\,\gevc. Different types of charged hadrons are distinguished using information from two ring-imaging Cherenkov detectors (\rich). Photons, electrons and hadrons are identified by a calorimeter system consisting of scintillating-pad and preshower detectors, an electromagnetic calorimeter and a hadronic calorimeter. Muons are identified by a system composed of alternating layers of iron and multiwire proportional chambers, and gas electron multiplier detectors. 

The online event selection is performed by a trigger~\cite{LHCb-DP-2012-004}, which consists of a hardware stage, based on information from the calorimeter and muon systems, followed by a software stage, which applies a full event reconstruction. Signal events considered in the analysis must fulfil hardware and software trigger requirements. At the hardware stage, at least one of the two following criteria must be satisfied: either a particle produced in the decay of the signal \Bm candidate leaves a deposit with high transverse energy in the hadronic calorimeter, or the event is accepted because particles not associated with the signal candidate fulfil other trigger requirements.
At the software stage, at least one charged particle should have a high \pt and a large \chisqip with respect to any PV, where \chisqip is defined as the difference in the vertex-fit \chisq of a given PV fitted with and without the considered track. A multivariate algorithm~\cite{BBDT} is used for the identification of secondary vertices that are consistent with the decay of a \bquark-hadron.
The software trigger designed to select \decay{\Bm}{\D\Kstarm} candidates requires a two-, three- or four-track secondary vertex with a large scalar sum of the \pt of the associated charged particles and a significant displacement from the PVs.
The PVs are fitted with and without the \B candidate tracks, and the PV that gives the smallest \chisqip is associated with the \B candidate.

The analysis presented is based on $pp$ collision data corresponding to an integrated luminosity of 1\invfb at centre-of-mass energy 7\tev, 2\invfb at 8\tev collected in 2011 and 2012 (Run 1) and 1.8\invfb at 13\tev collected in 2015 and 2016 (Run 2). There are a number of differences between data collected in Run 2 and Run 1. The main difference is the energy of the collisions which increases the $b\bar{b}$ production cross-section~\cite{LHCb-PAPER-2015-037}. The average number of $pp$ interactions per bunch crossing is reduced in Run 2 in comparison with Run 1. The net effect therefore is that, despite the higher energy of the collision, the background levels and signal-to-background ratios in Run 1 and Run 2 for the type of decay analysed here are similar. Before the start of Run 2, the aerogel radiator was removed from the first \rich detector~\cite{LHCb-DP-2012-003}, which improves its resolution. Hence, for momenta typical of decays in this analysis, the PID selection criteria have resulted in an increased efficiency of signal selection while simultaneously decreasing the rate of misidentified backgrounds. For this decay mode, the combination of higher $b\bar{b}$ production cross section, improved particle identification and improved online selection have resulted in the rate of selected signal candidates in data increasing by a factor 3 per unit integrated luminosity.
