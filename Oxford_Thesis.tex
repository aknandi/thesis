%%%%%%%%%%%%%%%%%%%%%%%%%%%%%%%%%%%%%%%%%%%%%%%%%%%%%%%%%%%%%%%
%% OXFORD THESIS TEMPLATE

% Use this template to produce a standard thesis that meets the Oxford University requirements for DPhil submission
%
% Originally by Keith A. Gillow (gillow@maths.ox.ac.uk), 1997
% Modified by Sam Evans (sam@samuelevansresearch.org), 2007
% Modified by John McManigle (mcmanigle@gmail.com), 2015

% I've (John) tried to comment this file extensively, so read through it to see how to use the various options.  Remember
% that in LaTeX, any line starting with a % is NOT executed.  Several places below, you have a choice of which line to use
% out of multiple options (eg draft vs final, for PDF vs for binding, etc.)  When you pick one, add a % to the beginning of
% the lines you don't want.


%%%%% CHOOSE PAGE LAYOUT
% The most common choices should be below.  You can also do other things, like replacing "a4paper" with "letterpaper", etc.

% This one will format for two-sided binding (ie left and right pages have mirror margins; blank pages inserted where needed):
\documentclass[a4paper,twoside]{ociamthesis}
% This one will format for one-sided binding (ie left margin > right margin; no extra blank pages):
%\documentclass[a4paper]{ociamthesis}
% This one will format for PDF output (ie equal margins, no extra blank pages):
%\documentclass[a4paper,nobind]{ociamthesis} 



%%%%% SELECT YOUR DRAFT OPTIONS
% Three options going on here; use in any combination.  But remember to turn the first two off before
% generating a PDF to send to the printer!

% This adds a "DRAFT" footer to every normal page.  (The first page of each chapter is not a "normal" page.)
\fancyfoot[C]{\emph{DRAFT Printed on \today}}  

% This highlights (in blue) corrections marked with (for words) \mccorrect{blah} or (for whole
% paragraphs) \begin{mccorrection} . . . \end{mccorrection}.  This can be useful for sending a PDF of
% your corrected thesis to your examiners for review.  Turn it off, and the blue disappears.
\correctionstrue


%%%%% BIBLIOGRAPHY SETUP
% Note that your bibliography will require some tweaking depending on your department, preferred format, etc.
% The options included below are just very basic "sciencey" and "humanitiesey" options to get started.
% If you've not used LaTeX before, I recommend reading a little about biblatex/biber and getting started with it.
% If you're already a LaTeX pro and are used to natbib or something, modify as necessary.
% Either way, you'll have to choose and configure an appropriate bibliography format...

% The science-type option: numerical in-text citation with references in order of appearance.
\usepackage[style=numeric-comp, sorting=none, backend=biber, doi=false, isbn=false]{biblatex}
\newcommand*{\bibtitle}{References}

% The humanities-type option: author-year in-text citation with an alphabetical works cited.
%\usepackage[style=authoryear, sorting=nyt, backend=biber, maxcitenames=2, useprefix, doi=false, isbn=false]{biblatex}
%\newcommand*{\bibtitle}{Works Cited}

% This makes the bibliography left-aligned (not 'justified') and slightly smaller font.
\renewcommand*{\bibfont}{\raggedright\small}

% Change this to the name of your .bib file (usually exported from a citation manager like Zotero or EndNote).
\addbibresource{references.bib}


% Uncomment this if you want equation numbers per section (2.3.12), instead of per chapter (2.18):
%\numberwithin{equation}{subsection}



%%%%% THESIS / TITLE PAGE INFORMATION
% Everybody needs to complete the following:
\title{Suitably impressive thesis title}
\author{Your Name}
\college{Your College}

% Master's candidates who require the alternate title page (with candidate number and word count)
% must also un-comment and complete the following three lines:
%\masterssubmissiontrue
%\candidateno{933516}
%\wordcount{28,815}

% Uncomment the following line if your degree also includes exams (eg most masters):
%\renewcommand{\submittedtext}{Submitted in partial completion of the}
% Your full degree name.  (But remember that DPhils aren't "in" anything.  They're just DPhils.)
\degree{Doctor of Philosophy}
% Term and year of submission, or date if your board requires (eg most masters)
\degreedate{Michaelmas 2014}


%%%%% YOUR OWN PERSONAL MACROS
% This is a good place to dump your own LaTeX macros as they come up.

% To make text superscripts shortcuts
	\renewcommand{\th}{\textsuperscript{th}} % ex: I won 4\th place
	\newcommand{\nd}{\textsuperscript{nd}}
	\renewcommand{\st}{\textsuperscript{st}}
	\newcommand{\rd}{\textsuperscript{rd}}



%%%%% THE ACTUAL DOCUMENT STARTS HERE
\begin{document}



%%%%% CHOOSE YOUR LINE SPACING HERE
% This is the official option.  Use it for your submission copy and library copy:
\setlength{\textbaselineskip}{22pt plus2pt}
% This is closer spacing (about 1.5-spaced) that you might prefer for your personal copies:
%\setlength{\textbaselineskip}{18pt plus2pt minus1pt}

% You can set the spacing here for the roman-numbered pages (acknowledgements, table of contents, etc.)
\setlength{\frontmatterbaselineskip}{17pt plus1pt minus1pt}

% Leave this line alone; it gets things started for the real document.
\setlength{\baselineskip}{\textbaselineskip}


%%%%% CHOOSE YOUR SECTION NUMBERING DEPTH HERE
% You have two choices.  First, how far down are sections numbered?  (Below that, they're named but
% don't get numbers.)  Second, what level of section appears in the table of contents?  These don't have
% to match: you can have numbered sections that don't show up in the ToC, or unnumbered sections that
% do.  Throughout, 0 = chapter; 1 = section; 2 = subsection; 3 = subsubsection, 4 = paragraph...

% The level that gets a number:
\setcounter{secnumdepth}{2}
% The level that shows up in the ToC:
\setcounter{tocdepth}{2}


%%%%% ABSTRACT SEPARATE
% This is used to create the separate, one-page abstract that you are required to hand into the Exam
% Schools.  You can comment it out to generate a PDF for printing or whatnot.
\begin{abstractseparate}
	%Your abstract text goes here.  Check your departmental regulations, but generally this should be less than 300 words. 

Measurements of \CP violation in \btodkst decays, including the sensitivity to the CKM angle \Pgamma, are presented in this thesis. Decays of the \Dz meson to \Km\pip, \Km\Kp, \pim\pip, \pim\Kp, \Km\pip\pim\pip, \pim\pip\pim\pip, and \pim\Kp\pim\pip are investigated. Observables are measured that are sensitive to the CKM angle \Pgamma and the hadronic parameters of the decay \rb and \deltab. These measurements are performed using proton-proton collision data collected by the Large Hadron Collider beauty (\lhcb) experiment from 2011 to 2016. These data correspond to the largest yields observed in these modes. The world's first \CP violation measurements in four-body \Dz decay modes using the \btodkst channel are presented. The first evidence of the \btodkst mode, where the \Dz decays to \pim\Kp, is also obtained with a signal significance of 4.2$\sigma$. A value of $\Pgamma = 40.9^{\circ}$ is measured with a $1\sigma$ confidence interval of $[25, 58]^{\circ}$, and $2\sigma$ confidence intervals of $[10, 87]^{\circ}$ and $[97,167]^{\circ}$, which is consistent with the world average measurement of \Pgamma. The world's most precise value of $\rb = 0.113^{+0.017}_{-0.019}$ for \btodkst decays is measured.
 % Create an abstract.tex file in the 'text' folder for your abstract.
\end{abstractseparate}


% JEM: Pages are roman numbered from here, though page numbers are invisible until ToC.  This is in
% keeping with most typesetting conventions.
\begin{romanpages}

% Title page is created here
\maketitle

%%%%% DEDICATION -- If you'd like one, un-comment the following.
%\begin{dedication}
%This thesis is dedicated to\\
%someone\\
%for some special reason\\
%\end{dedication}

%%%%% ACKNOWLEDGEMENTS -- Nothing to do here except comment out if you don't want it.
\begin{acknowledgements}
 	\subsection*{Personal}

This is where you thank your advisor, colleagues, and family and friends.

Lorem ipsum dolor sit amet, consectetur adipiscing elit. Vestibulum feugiat et est at accumsan. Praesent sed elit mattis, congue mi sed, porta ipsum. In non ullamcorper lacus. Quisque volutpat tempus ligula ac ultricies. Nam sed erat feugiat, elementum dolor sed, elementum neque. Aliquam eu iaculis est, a sollicitudin augue. Cras id lorem vel purus posuere tempor. Proin tincidunt, sapien non dictum aliquam, ex odio ornare mauris, ultrices viverra nisi magna in lacus. Fusce aliquet molestie massa, ut fringilla purus rutrum consectetur. Nam non nunc tincidunt, rutrum dui sit amet, ornare nunc. Donec cursus tortor vel odio molestie dignissim. Vivamus id mi erat. Duis porttitor diam tempor rutrum porttitor. Lorem ipsum dolor sit amet, consectetur adipiscing elit. Sed condimentum venenatis consectetur. Lorem ipsum dolor sit amet, consectetur adipiscing elit.

Aenean sit amet lectus nec tellus viverra ultrices vitae commodo nunc. Mauris at maximus arcu. Aliquam varius congue orci et ultrices. In non ipsum vel est scelerisque efficitur in at augue. Nullam rhoncus orci velit. Duis ultricies accumsan feugiat. Etiam consectetur ornare velit et eleifend.

Suspendisse sed enim lacinia, pharetra neque ac, ultricies urna. Phasellus sit amet cursus purus. Quisque non odio libero. Etiam iaculis odio a ex volutpat, eget pulvinar augue mollis. Mauris nibh lorem, mollis quis semper quis, consequat nec metus. Etiam dolor mi, cursus a ipsum aliquam, eleifend venenatis ipsum. Maecenas tempus, nibh eget scelerisque feugiat, leo nibh lobortis diam, id laoreet purus dolor eu mauris. Pellentesque habitant morbi tristique senectus et netus et malesuada fames ac turpis egestas. Nulla eget tortor eu arcu sagittis euismod fermentum id neque. In sit amet justo ligula. Donec rutrum ex a aliquet egestas.

\subsection*{Institutional}

If you want to separate out your thanks for funding and institutional support, I don't think there's any rule against it.  Of course, you could also just remove the subsections and do one big traditional acknowledgement section.

Lorem ipsum dolor sit amet, consectetur adipiscing elit. Ut luctus tempor ex at pretium. Sed varius, mauris at dapibus lobortis, elit purus tempor neque, facilisis sollicitudin felis nunc a urna. Morbi mattis ante non augue blandit pulvinar. Quisque nec euismod mauris. Nulla et tellus eu nibh auctor malesuada quis imperdiet quam. Sed eget tincidunt velit. Cras molestie sem ipsum, at faucibus quam mattis vel. Quisque vel placerat orci, id tempor urna. Vivamus mollis, neque in aliquam consequat, dui sem volutpat lorem, sit amet tempor ipsum felis eget ante. Integer lacinia nulla vitae felis vulputate, at tincidunt ligula maximus. Aenean venenatis dolor ante, euismod ultrices nibh mollis ac. Ut malesuada aliquam urna, ac interdum magna malesuada posuere.
\end{acknowledgements}

%%%%% ABSTRACT -- Nothing to do here except comment out if you don't want it.
\begin{abstract}
	%Your abstract text goes here.  Check your departmental regulations, but generally this should be less than 300 words. 

Measurements of \CP violation in \btodkst decays, including the sensitivity to the CKM angle \Pgamma, are presented in this thesis. Decays of the \Dz meson to \Km\pip, \Km\Kp, \pim\pip, \pim\Kp, \Km\pip\pim\pip, \pim\pip\pim\pip, and \pim\Kp\pim\pip are investigated. Observables are measured that are sensitive to the CKM angle \Pgamma and the hadronic parameters of the decay \rb and \deltab. These measurements are performed using proton-proton collision data collected by the Large Hadron Collider beauty (\lhcb) experiment from 2011 to 2016. These data correspond to the largest yields observed in these modes. The world's first \CP violation measurements in four-body \Dz decay modes using the \btodkst channel are presented. The first evidence of the \btodkst mode, where the \Dz decays to \pim\Kp, is also obtained with a signal significance of 4.2$\sigma$. A value of $\Pgamma = 40.9^{\circ}$ is measured with a $1\sigma$ confidence interval of $[25, 58]^{\circ}$, and $2\sigma$ confidence intervals of $[10, 87]^{\circ}$ and $[97,167]^{\circ}$, which is consistent with the world average measurement of \Pgamma. The world's most precise value of $\rb = 0.113^{+0.017}_{-0.019}$ for \btodkst decays is measured.

\end{abstract}

%%%%% MINI TABLES
% This lays the groundwork for per-chapter, mini tables of contents.  Comment the following line
% (and remove \minitoc from the chapter files) if you don't want this.  Un-comment either of the
% next two lines if you want a per-chapter list of figures or tables.
\dominitoc % include a mini table of contents
%\dominilof  % include a mini list of figures
%\dominilot  % include a mini list of tables

% This aligns the bottom of the text of each page.  It generally makes things look better.
\flushbottom

% This is where the whole-document ToC appears:
\tableofcontents

\listoffigures
	\mtcaddchapter
% \mtcaddchapter is needed when adding a non-chapter (but chapter-like) entity to avoid confusing minitoc

% Uncomment to generate a list of tables:
%\listoftables
%	\mtcaddchapter

%%%%% LIST OF ABBREVIATIONS
% This example includes a list of abbreviations.  Look at text/abbreviations.tex to see how that file is
% formatted.  The template can handle any kind of list though, so this might be a good place for a
% glossary, etc.
% First parameter can be changed eg to "Glossary" or something.
% Second parameter is the max length of bold terms.
\begin{mclistof}{List of Abbreviations}{3.2cm}

\item[1-D, 2-D] One- or two-dimensional, referring in this thesis to spatial dimensions in an image.

\item[Otter] One of the finest of water mammals.

\item[Hedgehog] Quite a nice prickly friend.

\end{mclistof} 


% The Roman pages, like the Roman Empire, must come to its inevitable close.
\end{romanpages}


%%%%% CHAPTERS
% Add or remove any chapters you'd like here, by file name (excluding '.tex'):
\flushbottom
%\begin{savequote}[8cm]
%\textlatin{Neque porro quisquam est qui dolorem ipsum quia dolor sit amet, consectetur, adipisci velit...}
%
%There is no one who loves pain itself, who seeks after it and wants to have it, simply because it is pain...
%  \qauthor{--- Cicero's \textit{de Finibus Bonorum et Malorum}}
%\end{savequote}

%The only true wisdom is knowing you know nothing - Socrates
%I'd rather have questions that I cannot answer than answers that I cannot question - Richard Feynmann
%Research is what I am doing when I don't know what I'm doing - Wernher von Braun
%Science is a way of thinking much more than it is a body of knowledge - Carl Sagan
%Two things are infinite: the universe and human stupidity; and I'm not sure about the universe - Albert Einstein
%There's nothing quite as frightening as someone who knows they are right - Michael Faraday

\chapter{\label{ch:1-intro}Introduction} 

The field of particle physics has arisen from curiosity and desire to understand the world around us. Throughout the last century a greater understanding of the most fundamental particles that constitute matter and the interactions between them has been developed, which has culminated in the Standard Model (SM) of particle physics. This elegant and well tested model represents our current best understanding of the Universe.

\CP violation enters the SM through a single phase in the CKM quark mixing matrix. The unitarity constraints of this matrix can be graphically represented by a triangle in the complex plane. One of the angles in this triangle, conventionally labelled \Pgamma, is a very useful parameterisation of \CP violation, due to the fact that it can be measured in purely tree-level processes. While processes that contain loops are expected to have sensitivity to physics beyond the SM, a tree-level measurement acts as a SM ``benchmark''. Therefore, by comparing tree-level SM measurements of \Pgamma to indirect, loop-level measurements, any deviation observed indicates new physics. Hence the uncertainties on direct measurements of \Pgamma must be minimised in order to make such comparisons meaningful.

This thesis investigates \CP violation in \btodkst decays, where \D is a superposition of \Dz and \Dzb decaying to two- and four-body final states containing charged pions and kaons. The sensitivity of these decays to the CKM angle \Pgamma is studied. The measurement involves isolating the \btodkst families of decays before measuring various observables relating to the yields, which are sensitive to \CP violation. Information on the CKM angle \Pgamma is extracted, along with other hadronic parameters of the decay.

The data used in this thesis are collected by the LHCb detector, located at the CERN Large Hadron Collider (LHC). The experiment specialises in \CP violation measurements of \bquark and \cquark hadrons. In total the data used in this thesis consists of 1\invfb and 2\invfb of $pp$ collisions at $\sqrt{s} = 7\tev \text{ and } 8\tev$ collected in 2011 and 2012 respectively, and 1.8\invfb at $\sqrt{s} = 13\tev$ collected in 2015 and 2016.

Chapter \ref{ch:2-background} covers the background theory behind \CP violation in the Standard Model. The unitarity triangle and CKM angle \Pgamma are discussed, followed by the methods to extract \CP violation measurements from \btodkst decays. Previous \Pgamma measurements using these decays are also summarised. 

In Chapter \ref{ch:3-detector} an overview of the \lhcb detector is presented, covering each of the sub-detectors: \velo, tracking systems, dipole magnet, \rich sub-detectors, electromagnetic and hadronic calorimeters and muon system. This chapter also introduces the trigger system and reconstruction, as well as discussing the process creating simulated event samples in the \lhcb detector.

The original research performed by the author is primarily documented
in Chapters \ref{ch:4-selection} - \ref{ch:6-interpretation}. Chapter \ref{ch:4-selection} presents a discussion of the selection requirements of the analysis including, reconstruction, particle identification, reducing so-called peaking backgrounds and the multivariate analysis techniques used. This chapter also presents the mass parameterisation of the favoured modes, covering the shapes for each of the model components, before revealing the full mass fits.

Chapter \ref{ch:5-cpfit} covers the simultaneous fit that is used to extract the \CP observables. The setup of the simultaneous fit is discussed in detail, before moving onto the results obtained from the fit and the systematic uncertainties. Finally, the results are interpreted in terms of the parameters of interest, \rb, \deltab and \Pgamma, in Chapter \ref{ch:6-interpretation}, where \rb and \deltab are the hadronic parameters of the \Bm-meson decay. Here the external inputs used in the interpretation are discussed, before providing results on the sensitivity to \rb, \deltab and \Pgamma in the \btodkst decay channel. Additionally, the expected future sensitivity to these parameters in \btodkst decays is discussed.

The work presented in this thesis was published in a single article in the Journal of High Energy Physics \textbf{11} (2017) 156. 




\minitoc



% $Id: introduction.tex 87303 2016-02-08 13:44:29Z lafferty $
\begin{savequote}[8cm]
\textlatin{Neque porro quisquam est qui dolorem ipsum quia dolor sit amet, consectetur, adipisci velit...}

There is no one who loves pain itself, who seeks after it and wants to have it, simply because it is pain...
  \qauthor{--- Cicero's \textit{de Finibus Bonorum et Malorum}}
\end{savequote}

\chapter{\label{ch:2-background}Background} 

\minitoc

\section{Introduction}
\label{sec:Introduction}

The CKM angle \Pgamma, defined as $\Pgamma \equiv arg\left(-\frac{\Vud{\Vub}^*}{\Vcd{\Vcb}^*}\right)$, is the least well known of the CKM angles. The latest \lhcb combination from direct measurements with charged and neutral \B decays and a variety for \D final states is $\left(70.9^{+7.1}_{-8.5}\right)^{\circ}$~\cite{LHCB-PAPER-2016-032-001}. Global fits to the CKM traingle from CKMfitter~\cite{CKMfitter2015} obtain a \Pgamma measurement of $(67.0^{+0.9}_{-2.0})^{\circ}$, where these uncertainties are driven by LQCD calculations. Therefore a degree level precison on a direct measurement of \Pgamma would test the consistency of these two measurements, which is an excellent test of the Standard Model. This can be achieved through a combination of many measurements of various \B and \D decays.

Direct measurements of \Pgamma can be made by exploiting the interference between the \decay{\bquark}{\cquark\uquarkbar\squark} and \decay{\bquark}{\uquark\cquarkbar\squark} transitions. These transitions are present in $B \to DK^{(*)}$ decays. This analysis measures \CP violation in \decay{\Bpm}{\D\Kstarpm} decays, where \D is either a \Dz or \Dzb, with D decays to 2 and 4 body final states. The interference is obtained by reconstructing the \Dz and \Dzb meson in the same final state, which leads to a measurement of \Pgamma.

Two analysis methods were used for 2 body \D decays. Decays of the \D to \CP (even) eigenstates \Kp\Km and \pip\pim using the GLW method~\cite{GL,GW} and decays to quasi-flavour eigenstates \Kpm\pimp (i.e the Cabibbo allowed decay \decay{\Dz}{\Km\pip} and the doubly Cabibbo supressed decay \decay{\Dz}{\pim\Kp}) using the ADS method~\cite{ADS,ADS-2001}. An ADS/GLW analysis has been published for \decay{\Bpm}{D\Kpm} and \decay{\Bz}{\D\Kstarz} at \lhcb~\cite{LHCb-PAPER-2016-003,LHCb-PAPER-2014-028}. The four body final states \decay{\Dz}{\Km\pip\pim\pip}, \decay{\Dz}{\pip\pim\pip\pim} \decay{\Dz}{\Kp\pim\pip\pim} are also investigated, which are included in the \decay{\Bpm}{D\Kpm} ADS/GLW analysis~\cite{LHCb-PAPER-2016-003}. These \Dz decays contain similar physics to the 2-body modes, so the GLW and ADS methods can be extended to these final states.

The \decay{\Bpm}{\D\Kstarpm} channel has previously been investigated at BaBar using a variety of D modes: non-\CP states $\Km\pip$, $\pim\Kp$, \CP even eigenstates $\Kp\Km$, $\pip\pim$ and \CP odd eigenstates $\KS\piz$, $\KS\phi$ and $\KS\omega$~\cite{BaBarDKstar}. The yield obtained in the favoured \decay{\Dz}{\Km\pip} mode in this analysis was $231 \pm 17$. In this BaBar analysis the non-resonant \decay{\B}{\D\KS\pi} contribution is considered negligible and therefore any effects of these non-resonant decays are ignored. Using yields in the \decay{\Bpm}{\D\Kpm} analysis~\cite{LHCb-PAPER-2016-003}, and estimated relative reconstruction/selection efficiencies and branching fraction it is expected that the Run 1 yields will be larger than those observed at BaBar. Therefore the analysis was motivated at this time. With the subsequent addition of a significant portion of Run 2 data this conclusion only holds stronger.

For the ADS decay, both favoured and supressed combinations of \Bm and \Dz decays are considered. The favoured \Bm decay is \decay{\Bm}{\Dz\Kstarm} and the suppressed is \decay{\Bm}{\Dzb\Kstarm}. Equivalently, the favoured \Dz decay is \decay{\Dz}{\Km\pip} and the supressed is \decay{\Dz}{\pim\Kp}. Therefore, the favoured combination has the \B meson and \kaon from the \D with the same sign: \decay{\Bm}{\D(\Km\pip)\Kstarm(\KS\pim)}, which is shortened to \decay{\Bm}{\D(\Km\pip)\Kstarm} in this note. The \D refers to a \Dz or \Dzb so this favoured combination also includes the suppressed \B decay followed by the suppressed \D decay. The supressed combination has the \B meson and \kaon from the \D with the opposite signs: \decay{\Bm}{\D(\Kp\pim)\Kstarm(\KS\pim)}, which is shortened to \decay{\Bm}{\D(\Kp\pim)\Kstarm} in this note. This is a combination of the suppressed \D decay followed by favoured \B decay and the favoured \B decay followed by the supressed \D decay. In this note the supressed ADS mode will be referred to as the ADS mode. These ideas can be extended to the 4 body modes with \decay{\Bm}{\D(\Km\pip\pim\pip)\Kstarm(\KS\pim)} and \decay{\Bm}{\D(\Kp\pim\pip\pim)\Kstarm(\KS\pim)} being analagous to \decay{\Bm}{\D(\Km\pip)\Kstarm(\KS\pim)} and \decay{\Bm}{\D(\Kp\pim)\Kstarm(\KS\pim)}.

Several physics observables are extracted in this analysis that relate to the physics parameters to be measured, namely \Pgamma, $r_B$, $\delta_B$ and $\kappa$. The parameter $r_B$ is the magnitude of the ratio between the supressed and favoured amplitudes, which governs the size of the interference between the two amplitudes and $\delta_B$ is the strong phase difference between these amplitudes. The coherence factor $\kappa$ is discussed in Section \ref{sec:interpretation:coherence}. The seven physics observables that are measured in this analysis are:

\begin{enumerate}
\item{The \CP asymmetry for the favoured mode
\begin{equation}
A_{\kaon\pi} = \frac{\Gamma\left(\decay{\Bm}{\D(\Km\pip)\Kstarm}\right) - \Gamma\left(\decay{\Bp}{\D(\Kp\pim)\Kstarp}\right)}{\Gamma\left(\decay{\Bm}{\D(\Km\pip)\Kstarm}\right) + \Gamma\left(\decay{\Bp}{\D(\Kp\pim)\Kstarp}\right)}
\label{eqn:Akpi}
\end{equation}
}
\item{The \CP asymmetry for the \decay{\D}{\Kp\Km} mode
\begin{equation}
A_{\kaon\kaon} = \frac{\Gamma\left(\decay{\Bm}{\D(\kaon\kaon)\Kstarm}\right) - \Gamma\left(\decay{\Bp}{\D(\kaon\kaon)\Kstarp}\right)}{\Gamma\left(\decay{\Bm}{\D(\kaon\kaon)\Kstarm}\right) + \Gamma\left(\decay{\Bp}{\D(\kaon\kaon)\Kstarp}\right)} = A_{CP+}
\label{eqn:Akk}
\end{equation}
}
\item{The \CP asymmetry for the \decay{\D}{\pip\pim} mode
\begin{equation}
A_{\pi\pi} = \frac{\Gamma\left(\decay{\Bm}{\D(\pi\pi)\Kstarm}\right) - \Gamma\left(\decay{\Bp}{\D(\pi\pi)\Kstarp}\right)}{\Gamma\left(\decay{\Bm}{\D(\pi\pi)\Kstarm}\right) + \Gamma\left(\decay{\Bp}{\D(\pi\pi)\Kstarp}\right)} = A_{CP+}
\label{eqn:Apipi}
\end{equation}
}
\item{The ratio of the \decay{\D}{\Kp\Km} over the favoured mode
\begin{equation}
R_{\kaon\kaon} = \frac{\Gamma\left(\decay{\Bm}{\D(\kaon\kaon)\Kstarm}\right) + \Gamma\left(\decay{\Bp}{\D(\kaon\kaon)\Kstarp}\right)}{\Gamma\left(\decay{\Bm}{\D(\Km\pip)\Kstarm}\right) + \Gamma\left(\decay{\Bp}{\D(\Kp\pip)\Kstarp}\right)} \times \frac{|BF(D^0 \to K^-\pi^+)|}{|BF(D^0 \to K^-K^+)|} = R_{CP+}
\label{eqn:Rkk}
\end{equation}
}
\item{The ratio of the \decay{\D}{\pip\pim} over the favoured mode
\begin{equation}
R_{\pi\pi} = \frac{\Gamma\left(\decay{\Bm}{\D(\pi\pi)\Kstarm}\right) + \Gamma\left(\decay{\Bp}{\D(\pi\pi)\Kstarp}\right)}{\Gamma\left(\decay{\Bm}{\D(\Km\pip)\Kstarm}\right) + \Gamma\left(\decay{\Bp}{\D(\Kp\pim)\Kstarp}\right)} \times \frac{|BF(D^0 \to K^-\pi^+)|}{|BF(D^0 \to \pi^-\pi^+)|} = R_{CP+}
\label{eqn:Rpipi}
\end{equation}
}
\item{The ratio of the ADS mode over the favoured mode for \Bp decays
\begin{equation}
R^+_{K\pi} = \frac{\Gamma\left(\decay{\Bp}{\D(\Km\pip)\Kstarp}\right)}{\Gamma\left(\decay{\Bp}{\D(\Kp\pim)\Kstarp}\right)}
\label{eqn:Rplus}
\end{equation}
}
\item{The ratio of the ADS mode over the favoured mode for \Bm decays
\begin{equation}
R^-_{K\pi} = \frac{\Gamma\left(\decay{\Bm}{\D(\Kp\pim)\Kstarm}\right)}{\Gamma\left(\decay{\Bm}{\D(\Km\pip)\Kstarm}\right)}
\label{eqn:Rminus}
\end{equation}
}
\item{The \CP asymmetry for the favoured \decay{\Dz}{\Km\pip\pim\pip} mode
\begin{equation}
A_{\kaon\pi\pi\pi} = \frac{\Gamma\left(\decay{\Bm}{\D(\Km\pip\pi\pi)\Kstarm}\right) - \Gamma\left(\decay{\Bp}{\D(\Kp\pim\pi\pi)\Kstarp}\right)}{\Gamma\left(\decay{\Bm}{\D(\Km\pip\pi\pi)\Kstarm}\right) + \Gamma\left(\decay{\Bp}{\D(\Kp\pim\pi\pi)\Kstarp}\right)}
\label{eqn:Akpipipi}
\end{equation}
}
\item{The \CP asymmetry for the \decay{\D}{\pip\pim\pip\pim} mode
\begin{equation}
A_{\pi\pi\pi\pi} = \frac{\Gamma\left(\decay{\Bm}{\D(\pi\pi\pi\pi)\Kstarm}\right) - \Gamma\left(\decay{\Bp}{\D(\pi\pi\pi\pi)\Kstarp}\right)}{\Gamma\left(\decay{\Bm}{\D(\pi\pi\pi\pi)\Kstarm}\right) + \Gamma\left(\decay{\Bp}{\D(\pi\pi\pi\pi)\Kstarp}\right)}
\label{eqn:Apipipipi}
\end{equation}
}
\item{The ratio of the \decay{\D}{\pip\pim\pip\pim} over the favoured mode
{\footnotesize
\begin{equation}
R_{\pi\pi\pi\pi} = \frac{\Gamma\left(\decay{\Bm}{\D(\pi\pi\pi\pi)\Kstarm}\right) + \Gamma\left(\decay{\Bp}{\D(\pi\pi\pi\pi)\Kstarp}\right)}{\Gamma\left(\decay{\Bm}{\D(\Km\pip\pi\pi)\Kstarm}\right) + \Gamma\left(\decay{\Bp}{\D(\Kp\pim\pi\pi)\Kstarp}\right)} \times \frac{|BF(D^0 \to \Km\pip\pi\pi)|}{|BF(D^0 \to \pim\pip\pi\pi)|}
\label{eqn:Rpipipipi}
\end{equation}}
}
\item{The ratio of the 4 body ADS mode over the 4 body favoured mode for \Bp decays
\begin{equation}
R^+_{K\pi\pi\pi} = \frac{\Gamma\left(\decay{\Bp}{\D(\Km\pip\pim\pip)\Kstarp}\right)}{\Gamma\left(\decay{\Bp}{\D(\Kp\pim\pip\pim)\Kstarp}\right)}
\label{eqn:Rplus4body}
\end{equation}
}
\item{The ratio of the 4 body ADS mode over the 4 body favoured mode for \Bm decays
\begin{equation}
R^-_{K\pi\pi\pi} = \frac{\Gamma\left(\decay{\Bm}{\D(\Kp\pim\pip\pim)\Kstarm}\right)}{\Gamma\left(\decay{\Bm}{\D(\Km\pip\pim\pip)\Kstarm}\right)}
\label{eqn:Rminus4body}
\end{equation}
}
\end{enumerate}

The observables $R^+_{K\pi}$ and $R^-_{K\pi}$ are being used for the ADS mode as opposed to the more usual $R_{ADS}$ and $A_{ADS}$. This was preferred due to their relatively low correlation. The same applies to $R^+_{K\pi\pi\pi}$ and $R^-_{K\pi\pi\pi}$ for the 4 body mode. Equations \ref{exp_Acp} - \ref{exp_Rpm4body} descibe the relationship between the physics observables measured in this analysis and the physics parameters \Pgamma, $r_B$, $\delta_B$ and $\kappa$.

\begin{equation}
A_{CP+} = \frac{2 \kappa r_B\sin\delta_B\sin\gamma}{1 + r_B^2 + 2 \kappa r_B\cos\delta_B\cos\gamma}
\label{exp_Acp}
\end{equation}

\begin{equation}
R_{CP+} = 1 + r_B^2 + 2 \kappa r_B\cos\delta_B\cos\gamma
\label{exp_Rcp}
\end{equation}

\begin{equation}
R^{\pm}_{K\pi} = \frac{r_B^2 + r_D^2 + 2\kappa r_B r_D \cos(\delta_B + \delta_D \pm \gamma)}{1 + r_B^2r_D^2 + 2\kappa r_B r_D \cos(\delta_B - \delta_D \pm \gamma)}
\label{exp_Rpm}
\end{equation}

\begin{equation}
A_{\pi\pi\pi\pi} = \frac{2 \kappa\kappa_{4\pi} r_B\sin\delta_B\sin\gamma}{1 + r_B^2 + 2 \kappa\kappa_{4\pi} r_B\cos\delta_B\cos\gamma}
\label{exp_A4pi}
\end{equation}

\begin{equation}
R_{\pi\pi\pi\pi} = 1 + r_B^2 + 2 \kappa\kappa_{4\pi} r_B\cos\delta_B\cos\gamma
\label{exp_R4pi}
\end{equation}

\begin{equation}
R^{\pm}_{K\pi\pi\pi} = \frac{r_B^2 + \left(r_D^{K3\pi}\right)^2 + 2\kappa r_B \kappa_{K3\pi} r_D^{K3\pi} \cos(\delta_B + \delta_D^{K3\pi} \pm \gamma)}{1 + \left(r_Br_D^{K3\pi}\right)^2 + 2\kappa r_B \kappa_{K3\pi} r_D^{K3\pi} \cos(\delta_B - \delta_D^{K3\pi} \pm \gamma)}
\label{exp_Rpm4body}
\end{equation}

%The selection for the ADS mode has been performed blinded, however it has been noted that the data has been looked at for a masters thesis at Annecy. In order to reduce further risk of bias the ADS mode is currently blinded: the extracted signal yields of candidates for \decay{\Bm}{\D(\Kp\pim)\Kstarm} and \decay{\Bp}{\D(\Km\pip)\Kstarp} are hidden during the first phase of this analysis.

In Section \ref{sec:selection} of the note the methods for selecting \decay{\Bpm}{\D\Kstarpm} candidates are detailed, the various MC and PID efficienies are presented in Section \ref{sec:mc} as well as comparisons between data and MC, Section \ref{sec:massfit} details the mass parameterisation of the $K\pi$ favoured mode including discussion of the partially reconstructed backgrounds, and Section \ref{sec:backgrounds} goes on to discuss other backgrounds considered in the analysis, but not included in the mass fit. Finally Sections \ref{sec:cpfit}, \ref{sec:systematics} and \ref{sec:interpretation} contain the details and results of the \CP fit, discussion of systematic uncertainties and interpretation of results.



%% APPENDICES %% 
% Starts lettered appendices, adds a heading in table of contents, and adds a
%    page that just says "Appendices" to signal the end of your main text.
\startappendices
% Add or remove any appendices you'd like here:
\begin{savequote}[8cm]
\textlatin{Cor animalium, fundamentum e\longs t vitæ, princeps omnium, Microco\longs mi Sol, a quo omnis vegetatio dependet, vigor omnis \& robur emanat.}

The heart of animals is the foundation of their life, the sovereign of everything within them, the sun of their microcosm, that upon which all growth depends, from which all power proceeds.
  \qauthor{--- William Harvey \cite{harvey_exercitatio_1628}}
\end{savequote}

\chapter{\label{app:1-cardiophys}Review of Cardiac Physiology and Electrophysiology}

\minitoc

Appendices are just like chapters.  Their sections and subsections get numbered and included in the table of contents; figures and equations and tables added up, etc.  Lorem ipsum dolor sit amet, consectetur adipiscing elit. Sed et dui sem. Aliquam dictum et ante ut semper. Donec sollicitudin sed quam at aliquet. Sed maximus diam elementum justo auctor, eget volutpat elit eleifend. Curabitur hendrerit ligula in erat feugiat, at rutrum risus suscipit. Pellentesque habitant morbi tristique senectus et netus et malesuada fames ac turpis egestas. Integer risus nulla, facilisis eget lacinia a, pretium mattis metus. Vestibulum aliquam varius ligula nec consectetur. Maecenas ac ipsum odio. Cras ac elit consequat, eleifend ipsum sodales, euismod nunc. Nam vitae tempor enim, sit amet eleifend nisi. Etiam at erat vel neque consequat.

\section{Anatomy}
\label{sec:anatomy}

Lorem ipsum dolor sit amet, consectetur adipiscing elit. Donec accumsan cursus neque. Pellentesque eget tempor turpis, quis malesuada dui. Proin egestas, sapien sit amet feugiat vulputate, nunc nibh mollis nunc, nec auctor turpis purus sed metus. Aenean consequat leo congue volutpat euismod. Vestibulum et vulputate nisl, at ultrices ligula. Cras pulvinar lacinia ipsum at bibendum. In ac augue ut ante mollis molestie in a arcu.

Etiam vitae quam sollicitudin, luctus tortor eu, efficitur nunc. Vestibulum maximus, ante quis consequat sagittis, augue velit luctus odio, in scelerisque arcu magna id diam. Proin et mauris congue magna auctor pretium id sit amet felis. Maecenas sit amet lorem ipsum. Proin a risus diam. Integer tempus eget est condimentum faucibus. Suspendisse sem metus, consequat vel ante eget, porttitor maximus dui. Nunc dapibus tincidunt enim, non aliquam diam vehicula sed. Proin vel felis ut quam porta tempor. Vestibulum elit mi, dictum eget augue non, volutpat imperdiet eros. Praesent ac egestas neque, et vehicula felis.

Pellentesque malesuada volutpat justo, id eleifend leo pharetra at. Pellentesque feugiat rutrum lobortis. Curabitur hendrerit erat porta massa tincidunt rutrum. Donec tincidunt facilisis luctus. Aliquam dapibus sodales consectetur. Suspendisse lacinia, ipsum sit amet elementum fermentum, nulla urna mattis erat, eu porta metus ipsum vel purus. Fusce eget sem nisl. Pellentesque dapibus, urna vitae tristique aliquam, purus leo gravida nunc, id faucibus ipsum magna aliquet ligula. Lorem ipsum dolor sit amet, consectetur adipiscing elit. Proin sem lacus, rutrum eget efficitur sed, aliquam vel augue. Aliquam ut eros vitae sem cursus ultrices ut ornare urna. Nullam tempor porta enim, in pellentesque arcu commodo quis. Interdum et malesuada fames ac ante ipsum primis in faucibus. Curabitur maximus orci purus, ut molestie turpis pellentesque ut.

Donec lacinia tristique ultricies. Proin dignissim risus ut dolor pulvinar mollis. Proin ac turpis vitae nibh finibus ullamcorper viverra quis felis. Mauris pellentesque neque diam, id feugiat diam vestibulum vitae. In suscipit dui eu libero ultrices, et sagittis nunc blandit. Aliquam at aliquet ex. Nullam molestie pulvinar ex vitae interdum. Praesent purus nunc, gravida id est consectetur, convallis elementum nulla. Praesent ex dolor, maximus eu facilisis at, viverra eget nulla. Donec ullamcorper ante nisi. Sed volutpat diam eros. Nullam egestas neque non tortor aliquet, sed pretium velit tincidunt. Aenean condimentum, est ac vestibulum mattis, quam augue congue augue, mattis ultrices nibh libero non ante. Lorem ipsum dolor sit amet, consectetur adipiscing elit.

Aenean volutpat eros tortor, non convallis sapien blandit et. Maecenas faucibus nulla a magna posuere commodo. Nullam laoreet ante a turpis laoreet malesuada. Phasellus in varius sem. Vestibulum sagittis nibh sed tincidunt blandit. Donec aliquam accumsan odio sit amet lacinia. Integer in tellus diam. Vivamus varius massa leo, vitae ullamcorper metus pulvinar sed. Maecenas nec lorem ornare, elementum est quis, gravida massa. Suspendisse volutpat odio ex, ac ultrices leo ultrices vel. Sed sed convallis ipsum. Pellentesque euismod a nulla sed rhoncus. Sed vehicula urna vitae mi aliquet, non sodales lacus ullamcorper. Duis mattis justo turpis, id tempus est tempus eu. Curabitur vitae hendrerit ligula.

Curabitur non pretium enim, in commodo ligula. Etiam commodo eget ligula a lacinia. Vestibulum laoreet ante tellus, vel congue sapien ornare in. Donec venenatis cursus velit vitae pulvinar. Pellentesque habitant morbi tristique senectus et netus et malesuada fames ac turpis egestas. Suspendisse in metus lectus. Pellentesque gravida dolor eget finibus imperdiet. Duis id molestie tortor. Mauris laoreet faucibus facilisis. Aliquam vitae dictum massa, sit amet dignissim lacus.

Fusce eleifend tellus id ex consequat maximus. Donec ultrices ex ut turpis ornare, non molestie mi placerat. Nulla sit amet auctor nunc, sit amet euismod elit. Phasellus risus tellus, condimentum a metus et, venenatis tristique urna. Cras mattis felis eget ipsum fermentum egestas. Ut augue odio, venenatis id convallis vel, congue quis augue. Maecenas sed maximus est, posuere aliquet tortor. Ut condimentum egestas nisi eu porttitor. Ut mi turpis, posuere id lorem vel, elementum tempor arcu.

Morbi nisl arcu, venenatis non metus ac, ullamcorper scelerisque justo. Nulla et accumsan lorem. Mauris aliquet dui sit amet libero aliquet, in ornare metus porttitor. Integer ultricies urna eu consequat ultrices. Maecenas a justo id purus ultricies posuere sed et quam. Cum sociis natoque penatibus et magnis dis parturient montes, nascetur ridiculus mus. Sed eleifend risus quis aliquet gravida. Nullam ac erat porta est bibendum dictum in a dolor. Nam eget turpis viverra, vulputate lectus eget, mattis ligula. Nam at tellus eget dui lobortis sodales et ut augue. In vestibulum diam eget mi cursus, ut tincidunt nulla pellentesque.

Aliquam erat volutpat. Sed ultrices massa id ex mattis bibendum. Nunc augue magna, ornare at aliquet gravida, vehicula sed lorem. Quisque lobortis ipsum eu posuere eleifend. Duis bibendum cursus viverra. Nam venenatis elit leo, vitae feugiat quam aliquet sed. Cras velit est, tempus ac lorem sed, pharetra lobortis ipsum. Donec suscipit gravida interdum. Nunc non finibus est. Nullam turpis elit, tempus non ante.

\section{Mechanical Cycle}

Lorem ipsum dolor sit amet, consectetur adipiscing elit. Aenean tellus est, suscipit sed facilisis quis, malesuada at ipsum. Nam tristique urna quis quam iaculis, et mattis orci pretium. Praesent euismod elit vel metus commodo ultrices. Vestibulum et tincidunt ex, in molestie ex. Donec ullamcorper sollicitudin accumsan. Etiam ac leo turpis. Duis a tortor felis. Nullam sollicitudin eu purus ac hendrerit. Nam hendrerit ligula libero, eget finibus orci bibendum a. Aenean ut ipsum magna.

Ut viverra, sapien sed accumsan blandit, nisi sem tempus tellus, at suscipit magna erat ornare nunc. Proin lacinia, nisi ut rutrum malesuada, nibh quam pellentesque nunc, sit amet consectetur purus felis ac tortor. Suspendisse lacinia ipsum eu sapien pellentesque mattis. Mauris ipsum nunc, placerat non diam vel, efficitur laoreet nunc. Sed lobortis, ipsum eget gravida facilisis, sem nulla viverra mi, in placerat eros sem viverra lacus. Aliquam porta aliquet diam vel commodo. Nulla facilisi. Duis erat libero, lobortis vel hendrerit vitae, sagittis id dui. Nulla pretium eros nec quam tincidunt, vel luctus mi aliquam. Integer imperdiet purus in est tristique venenatis. Ut pellentesque, nunc vitae iaculis ultricies, urna turpis dignissim risus, a laoreet felis magna nec erat.

Quisque sollicitudin faucibus ligula, et egestas nibh dictum sit amet. Proin eu mi a lectus congue pretium eu quis arcu. Suspendisse vehicula libero eu ipsum aliquam, vel elementum nibh mattis. Sed sed sapien vitae turpis tristique pulvinar a ut metus. Etiam semper gravida est, mollis gravida est porta ac. Proin eget tincidunt erat. Maecenas ultrices erat eget purus ultricies, ut lacinia arcu dictum. Nam et nisi sit amet ex congue mattis vel eget lorem. Aliquam erat volutpat. Pellentesque porttitor nibh vitae elementum consectetur. Aenean et est lobortis, congue sapien non, ullamcorper sapien. Ut facilisis sem non dapibus vehicula.

Mauris euismod odio dolor, sit amet gravida mauris placerat et. Curabitur nec dolor non nibh molestie lobortis dignissim non ante. Nullam rutrum lobortis ultrices. Aenean ex erat, elementum sed maximus id, posuere id quam. Proin rutrum ex elit, pretium aliquam risus finibus at. Aenean egestas orci velit, sed aliquet sapien condimentum a. Duis consequat, arcu eu viverra venenatis, dolor lorem gravida lectus, non aliquet nisi sem at augue. Donec laoreet blandit luctus. Aenean vehicula nisl vel faucibus luctus. Sed ut semper velit, vitae laoreet magna. Sed at interdum magna.

Sed iaculis faucibus odio, eu aliquam purus efficitur vel. Cras at nulla ac enim congue varius ut et nulla. Integer blandit mattis augue.

\section{Electrical Cycle}
\label{sec:electcycle}

Lorem ipsum dolor sit amet, consectetur adipiscing elit. In faucibus condimentum rhoncus. Ut dictum nisl id risus gravida lobortis. Sed vehicula mollis tellus ut varius. Fusce eget egestas dui, et commodo dui. Proin sollicitudin interdum tempus. Nullam in elit a enim fringilla bibendum. Vestibulum sodales pellentesque condimentum. Nulla facilisi. Nunc et dolor in nulla eleifend dictum at vel ligula. Aliquam ut velit non elit ullamcorper porta ac et ex. Fusce ornare magna non nunc vestibulum, eget molestie quam dictum. In interdum aliquam odio, in posuere tellus convallis quis. Curabitur non diam elit. Proin vulputate orci diam, a tincidunt ante luctus eu. Ut a viverra ligula. Curabitur pulvinar tempus tellus eget suscipit.

Aliquam posuere massa at ante dapibus congue. Curabitur ullamcorper tortor eget consectetur aliquet. Mauris tempor magna id mauris fringilla, a varius erat blandit. Nam eleifend ullamcorper placerat. Phasellus augue tortor, volutpat bibendum lorem nec, fringilla volutpat nisl. Mauris cursus urna metus, vel eleifend orci iaculis ut. Sed sit amet scelerisque massa, quis consequat dui. Donec semper sem dui, ac placerat velit egestas vel. Nulla facilisi. Quisque tellus eros, sagittis malesuada augue ut, faucibus dictum nulla. Vestibulum non dapibus erat, ut consequat libero. Ut turpis mi, dapibus commodo libero lobortis, maximus vestibulum lectus. Vestibulum sit amet sapien dapibus, tincidunt leo in, suscipit arcu. Sed in erat bibendum, laoreet eros eu, pellentesque justo. Nulla sodales purus neque, eget maximus ipsum consequat at. Maecenas a nisl sagittis, tempus ipsum sed, dictum mauris.

Suspendisse posuere odio lacus, at auctor tortor vehicula sed. Phasellus suscipit ornare enim vitae placerat. Sed viverra purus vel sapien tempor, quis iaculis erat laoreet. Aenean vel nunc vestibulum, ornare nunc ac, mollis urna. Aenean ultrices felis ipsum, ac semper est ullamcorper in. Donec in justo varius, egestas tortor ut, venenatis augue. Duis mattis, ligula quis lacinia fringilla, tellus neque accumsan ipsum, vitae tempor metus elit vel nibh. Curabitur porttitor urna nec sapien tempor, et porttitor velit malesuada.

Suspendisse aliquam nisl quis placerat vulputate. Proin dapibus ipsum ac ante sagittis, volutpat auctor sem dapibus. Nam in facilisis odio. Integer ante mauris, eleifend et pulvinar in, venenatis quis ligula. Phasellus posuere sollicitudin tortor eget euismod. Maecenas mollis tortor eget justo vulputate sagittis. Etiam hendrerit massa quis ex molestie sodales. Quisque facilisis erat lacus, id convallis sem suscipit bibendum. Integer dui urna, pharetra sed porta sed, bibendum ut odio. Donec placerat at lectus egestas consequat. Sed id rhoncus est, vitae vulputate sapien. Fusce tempus quam lorem, id ornare turpis sodales sed. Integer aliquet urna eget condimentum consequat. Vestibulum quis dui vel ligula posuere luctus id nec turpis.

Nam vitae placerat lacus. Mauris scelerisque interdum volutpat. Nunc aliquet tristique enim, sit amet molestie felis ullamcorper vitae. Nullam sollicitudin orci orci, in condimentum tellus consectetur in. Nam id justo justo. Fusce eget finibus est. Proin id tortor nec quam cursus vehicula. Aliquam ultrices eros eros, a tincidunt elit eleifend auctor.

Nullam consectetur dapibus ligula sit amet efficitur. Nunc non posuere sapien. Vivamus dui nisl, aliquam id ipsum non, pulvinar ornare neque. Nunc rhoncus pretium congue. Fusce id laoreet enim. Cras sed massa in eros bibendum auctor in nec sem. Nam commodo, velit id porta consequat, mi arcu gravida lorem, ut aliquam elit ante quis dui. Quisque in massa sed nibh blandit dictum.

Vestibulum molestie consectetur porttitor. Donec tincidunt vel orci at pharetra. Nullam id felis sit amet nulla tempus lacinia. Integer egestas ullamcorper massa, ut ultricies diam congue sit amet. Cras sit amet velit at nibh vehicula finibus a et lorem. Cras odio metus, venenatis ut ultrices non, ornare ac orci. Morbi et nulla dui. Mauris dictum molestie nibh, eu efficitur lorem accumsan quis.

\section{Cellular Electromechanical Coupling}
\label{sec:electromech}

Lorem ipsum dolor sit amet, consectetur adipiscing elit. Nullam vitae consectetur metus, ac maximus ex. Quisque vitae ex eu lectus ultricies consequat vel non lorem. Etiam odio ipsum, tempus ut lobortis in, molestie ac leo. Vivamus mollis feugiat bibendum. Vestibulum eget venenatis quam. Aenean faucibus, massa sed ullamcorper porta, arcu nunc iaculis velit, quis consectetur purus neque placerat nibh. Vestibulum elit nunc, dignissim vulputate venenatis et, sodales non massa. Proin leo ligula, vehicula eu aliquam varius, posuere a dolor. Donec iaculis auctor neque, sit amet gravida libero porta vel. Vivamus consequat elementum lacus, at bibendum mauris egestas nec. Fusce fermentum diam eu dolor ornare, vitae vestibulum leo interdum. Morbi luctus libero quis dictum laoreet. Etiam semper porta ante, vel ullamcorper enim sodales quis.

Nullam eu nisi faucibus, fermentum ex auctor, tempor arcu. Phasellus condimentum erat mi, condimentum malesuada ligula congue venenatis. Nullam gravida imperdiet urna quis cursus. Ut tempus nec purus eget posuere. Cras non nulla sit amet justo aliquet pellentesque nec sed eros. Nam aliquam nisl urna, in placerat magna gravida venenatis. Donec interdum vel magna ullamcorper molestie. Nunc felis neque, rhoncus fringilla faucibus sit amet, ultrices sed magna. Maecenas malesuada hendrerit diam in ultrices. Nam libero urna, volutpat ut auctor eget, interdum sed odio. Vestibulum suscipit mauris nec augue ornare, ut eleifend nulla gravida. Proin imperdiet, mauris quis consectetur porta, leo dui convallis leo, id lobortis massa diam eu libero. Aenean hendrerit vel ante aliquam venenatis. Pellentesque bibendum pretium odio, ut sagittis lectus feugiat a. Donec porttitor vulputate lacus.

Nunc volutpat efficitur lacus in aliquet. Nullam non iaculis diam, at ultrices diam. Proin vehicula vulputate cursus. Morbi tempus sapien id urna lobortis interdum. Maecenas elementum sagittis elementum. Donec at sodales velit, a posuere tortor. Nulla id hendrerit tortor. Sed semper velit in magna sagittis pulvinar. Nulla nec arcu molestie, ultricies sapien sit amet, sollicitudin nisi. Donec nisi massa, suscipit ut dignissim quis, lacinia id leo.

Suspendisse ut mi metus. Morbi tincidunt ligula in porttitor consectetur. Integer eu urna urna. Suspendisse potenti. Mauris sit amet felis eu diam auctor ullamcorper. Morbi in porta nisi. Nam ante tortor, venenatis vitae tempor sed, sagittis vitae velit. In semper orci sit amet nisi ullamcorper varius. Aenean dignissim ultrices imperdiet. Maecenas lacinia enim id neque porttitor iaculis. Curabitur laoreet ante ut urna dignissim, id sollicitudin metus consectetur. Aenean massa ipsum, auctor vel ante vel, blandit dignissim libero. Fusce interdum ac magna et interdum.



%%%%% REFERENCES

% JEM: Quote for the top of references (just like a chapter quote if you're using them).  Comment to skip.
\begin{savequote}[8cm]
The first kind of intellectual and artistic personality belongs to the hedgehogs, the second to the foxes \dots
  \qauthor{--- Sir Isaiah Berlin \cite{berlin_hedgehog_2013}}
\end{savequote}

\setlength{\baselineskip}{0pt} % JEM: Single-space References

{\renewcommand*\MakeUppercase[1]{#1}%
\printbibliography[heading=bibintoc,title={\bibtitle}]}


\end{document}
